\documentclass[12pt,letterpaper]{article}
\usepackage{epsfig}
 \setlength{\textwidth}{6.5in}
 \setlength{\textheight}{8.5in}
 \setlength{\evensidemargin}{-0.15in}
 \setlength{\oddsidemargin}{-0.15in}
 \setlength{\topmargin}{0.0in}
 \setlength{\parskip}{2mm}
 \bibliographystyle{unsrt}
\title{\large EPANET Backtracking Extension (BTX)\\User's Manual (Version 1.0)}
\author{\normalsize Feng Shang, James G. Uber\\
\small Department of Civil and Environmental Engineering \\
\small University of Cincinnati, P.O. Box 210071, Cincinnati, Ohio
45221-0071}
%\date{}
\begin{document}
\maketitle
\section*{Introduction}
Various computer models have been developed to simulate the change
of disinfectant species concentration within water distribution
systems. These models can be divided into two general categories
\cite{Rossman96}: Eulerian and Lagrangian. Eulerian models divide
the network into a series of fixed control elements and record the
changes at the boundaries and within these elements, while
Lagrangian models track changes of discrete parcels of water as they
travel through the network. Traditional models are efficient in
modeling water quality at all nodes over time, but forget the
internal details of the links between water qualities at upstream
and downstream locations (or, at least, they are not available in a
useful form). These internal details - the input-output (I/O)
information - include the number of flow paths, their time delays,
and their impact on water quality. Such I/O information is critical
to certain applications. One such example is the design of feedback
control algorithms used to regulate disinfectant injections based on
measurements at a distributed set of sensor locations
\cite{wzicc00}. Such an application assumes that disinfectant
injection can be related to output concentration, using either an
I/O or state space model. For water distribution systems, the state
space dimension can preclude real-time use of a state space model
for computational reasons. Input-output information is also
important to determine the location and time of a contaminant
intrusion, accidental or intentional, into the water distribution
systems. When water quality contamination is detected at a
particular location, fault diagnosis should focus on areas that are
hydraulically connected to the problem location. More generally, I/O
information is required to explain water quality transformation
processes that are {\it path-dependent}, such as those that involve
pipe material interactions.

Zierolf et al. \cite{ZPU98} first developed an I/O model for
chlorine transport in networks without storage tanks. Their
algorithm tracks the travel of water parcels in networks in reverse
time. This model can find all the paths from the input to the output
and the corresponding delays and was used to calibrate pipe chlorine
reaction rates off-line. The particle backtrack algorithm (PBA)
\cite{shang2002} extends the work of Zierolf et al. to consider the
existence of tanks and to allow multiple water sources and quality
inputs. Unlike Zierolf et al., who track each parcel individually
and sequentially from output to input, the PBA tracks simultaneously
a large number of water parcels, moving them simultaneously along
their paths (in reverse time). This change results in a simpler
algorithm to describe, and one that is likely more efficient because
the algorithm moves in parallel with the time variation in the
hydraulic solution.

The potential research and engineering applications motivate the
extension of EPANET \cite{Rossman2000} to include PBA so that user
can have easy access to the detailed I/O information that otherwise
are not convenient to obtain.
\section*{Overview of Particle Backtracking Algorithm}
Under the assumption of first order reaction, water quality, such as
chlorine concentration, at a downstream location (output) can be
expressed as a linear combination of water quality strength
(concentration, mass input rate, or strength of secondary water
quality input such as booster chlorination) at some upstream
locations and time (inputs), weighted by water quality impact
coefficients that reflect water dilution at junctions and storage
tanks and water quality change due to reaction. Mathematically, such
linear relationship can be written as:
\begin{equation}\label{lineario}
c_o(T_o)=\sum_{j=1}^N\gamma_js_{j}(T_o-t_j)
\end{equation}
where $N=$ number of flow paths between water quality inputs and
output; $j=$ flow path index; $T_o=$ output time; $s_{j}=$ water
quality input strength for path $j$; $t_j=$ water travel time or
time delay for flow path $j$; and $\gamma_j=$ water quality impact
coefficient. The impact coefficient $\gamma_j$ is the sensitivity of
output concentration (at time $T_o) $to input strength (at time
$T_o-t_j$), of which the unit depends on the type of water quality
input. Value of the impact coefficients represents the decay or
growth due to chemical reaction and dilution at the network
junctions and storage tanks. It should be noted that the $N$ inputs
must account for all the water the output receives, otherwise
Equation \ref{lineario} is not valid. The number of inputs, $N$, can
be larger than the number of water quality input locations because
it is possible that there exist multiple flow paths between an input
location and an output and these flow paths are associated with
different time delays

Particle backtracking algorithm tracks the water reaching the output
until all the water parcels are traced back to either true water
source or booster water quality source that fixes the concentration
of any flow leaving such ``set point'' source. Backtracking stops at
the ``set point'' booster source because the output concentration is
not sensitive to water quality input upstream of the ``set point''
water quality input.

Sometime it is interesting to know what is the impact of constant
water quality input during a certain period of time on output water
quality. The aggregated impact of an water quality source during a
period of time can be measured with the sum of impact coefficients.
If a water quality output at time $T_o$ is hydraulically connected
to a source location with constant strength, $s$, at $n$ different
times $T<t_1<t_2\cdots<t_n<T+\delta t$, the output water quality can
be calculated as:
\begin{equation}\label{lineario2}
c_o(T_o)=\sum_{j=1}^n(\gamma_js_j(t_j))=\sum_{j=1}^n(\gamma_j)s=\gamma
s
\end{equation}
where $\gamma$ is the aggregated impact coefficient of the source
between $T$ and $T+\delta T$, with the assumption of constant water
quality strength during that period. It is the sum of the impact
coefficients of multiple flow paths with time delays between $T$ and
$T+\delta T$ from the source to the output.

EPANET-BTX is a software tool through which users can have access to
the information that can be obtained with particle backtracking
algorithm, which includes number of flow paths, time delay and
impact coefficient associated with flow paths, and aggregated impact
coefficients of a source (during a certain period) on specific water
quality output. EPANET-BTX can only track a single species that
follows the first order reaction kinetics, i.e.
\begin{equation}
\frac{dc}{dt}=kc
\end{equation}
where $c=$ the concentration of the component to be tracked and $k=$
first order rate coefficient. The reaction rate coefficient $k$,
which can be zero for non reactive constituent, depends on both the
bulk reaction and wall reaction coefficients, which are specified in
the EPANET input file. The details of EPANET input file are in
\cite{Rossman2000}.
\section*{Program Usage}
EPANET-BTX is distributed as a compressed archive file named
epanetbtx.zip which contains the following files:
\begin{itemize}
\item epanetbtx.dll:    EPANET-BTX dynamic linkage library that should be used in conjunction with the EPANET Toolkit library (epanet2.dll).
\item epanetbtx.lib:    Microsoft C/C++ library file for
epanetbtx.dll.
\item epanetbtx.h:  C/C++ header file for EPANET-BTX.
\item epanetbtx.pdf:    EPANET-BTX users manual.
\item epanet2.h:    header file for EPANET toolkit.
\item epanet2.dll: dynamic linkage library of EPANET toolkit.
\item epanet2.lib: Microsoft C/C++ library file for
epanet2.dll.
\end{itemize}
\subsection*{Example Toolkit Usage}
EPANET-BTX function library must be used in conjunction with the
EPANET Programmer's Toolkit which is slightly modified from the
original one. EPANET toolkit functions are called to open EPANET
input file, run hydraulic simulation and save hydraulic simulation
results for backtracking modeling. Application can be written in any
programming language that can call Windows DLL functions. The
following is an example that is written in C language and shows how
EPANET-BTX toolkit function can be used. The next section lists all
the functions included in the BTX toolkit.

The program includes two header files: \texttt{epanet2.h} and
\texttt{epanetbtx.h} that are C/C++ header files for EPANET
Programmer's Toolkit and Backtracking Extension Tookit,
respectively. The program begins with the EPANET Programmer Toolkit
function \texttt{ENopen} to open an EPANET input file and read in
all the network data. Other EPANET Programmer's Tookit functions can
be called after \texttt{ENopen} to obtain and change network data,
such as simulation duration, hydraulic time step, bulk decay
coefficient, etc. In the example program, EPANET Tooklit function
\texttt{ENsettimeparam} is called to set the simulation length to be
240 hours. \texttt{ENsolveH} and \texttt{ENsavehydfile} are then
called to solve network hydraulics and save the hydraulic solution
in a binary file. The function \texttt{BTXopen} is used to associate
hydraulics data with backtracking modeling by specifying the name of
a hydraulic binary file. If a binary hydraulic file is available
from some early application, functions \texttt{ENsolveH} and
\texttt{ENsavehydfile} are not needed. \texttt{BTXopen} is followed
by function \texttt{BTXinit}, which allocates the memory for
backtracking modeling and reads in hydraulic data from the binary
hydraulic file specified with \texttt{BTXopen}. The function
\texttt{BTXinit} has two input arguments: the first is the time step
in hours for impact coefficient aggregation and the second is the
impact coefficient thresh hold to delete backtracking particle with
small enough impact on output water quality. Function
\texttt{BTXsetinput} can be used to specify the nodes whose impacts
on output water quality are of interests to the modeler, while
\texttt{BTXsetinputstrength} is used to set the source strength of a
true water quality source. In the example, a constant source
strength of 2.0 $mg/L$ is assumed for the node with ID ``sourceid''
(``10''). A node must be set to be an ``input'' before source
strength is assigned, but an input is not necessarily a true water
quality source. \texttt{BTXsim} is the function to be called for
backtracking modeling and the two input arguments are output node id
and water quality simulation time. After backtracking modeling for
node with ID ``outputid'' (``32'') at ``sampletime'',
\texttt{BTXout} is called to calculate the output water quality
given water quality source strength set by
\texttt{BTXsetinputstrengh} and \texttt{BTXgetimpact} is called to
withdraw the vector of impact coefficients quantifying the water
quality impact of the node ``sourceid'' on node ``outputid'' at
``sampletime''. The first argument of the \texttt{BTXgetimpact} must
be a valid node id that has been specified as an input using
\texttt{BTXsetinput}. BTXgetimpact writes the impact coefficients of
``sourceid'' on ``outputid'' at ``sampletime'' into the
``impactvector'' with length ``impactvectorlength'' : because the
impact time step is set to be 1.0 hour, impactvector[n] contains
aggregated impact coefficient of constant water quality source at
``sourceid'' between hour n and n+1 on ``outputid'' at
``sampletime''. If impactvector[n] returns value larger than 0, node
``sourceid'' between hour n and n+1 is hydraulically connected to
node ``outputid'' at ``samlpetime'', i.e. part of water node
``outputud'' receives at ``samlpetime'' pass through node
``soueceid'' sometime between hour n and n+1. Function
\texttt{BTXgetpathnumber} obtains the number of flow paths
(backtracked), which starts at ``outputid'' and ends at either water
source or SETPOINT secondary water quality source. Also available
are toolkit functions to get detailed flow path information, such as
time delay and impact coefficient, and to check if water flows
through a specific pipe before reaching the output (see next section
for details). The functions \texttt{BTXsim}, \texttt{BTXout},
\texttt{BTXgetimpact} and functions to obtain flow path information
can be called repeatedly to do backtracking modeling and flow path
analysis for any node and any time. At the end of the program,
function \texttt{BTXclose} and EPANET Toolkit function
\texttt{ENclose} are called in sequence to free the memories used by
backtracking application and EPANET.
\\\\
\#include``epanetbtx.h''\\\#include ``epanet2.h''\\\\int main()\\\{\\
char *epainpfile=``net1.inp'', *eparptfile=``rpt'', *outputid=``32'', *sourceid = ``10'';\\
float sampletime = 200.0;\\
float impactvector[224];\\
float sourcestrength = 2.0, sourcepatstep = 1.0, samplestep = 1.0, impactstep = 1.0, thresh = 0.000001, wqres;\\
int i, sourcepatlength=1, impactvectorlength = 224, pathnumber;\\
\\ENopen(epainpfile, eparptfile, ``'');\\
ENsettimeparam(EN\_DURATION, 240*3600);\\
ENsolveH();\\
ENsavehydfile(``hyddata'');\\
BTXopen(``hyddata'');\\
BTXinit(impactstep, thresh);\\
BTXsetinput(sourceid, EN\_SETPOINT);\\
BTXsetinputstrength(sourceid, \&sourcestrength, sourcepatlength, sourcepatstep);\\
for( i = 0; i $<$ 24; i++)\\\{\\
sampletime = sampletime + samplestep*i;\\
BTXsim(outputid, sampletime);\\
BTXout(\&wqres);\\
BTXgetimpact(sourceid, impactvector, impactvectorlength);\\
BTXgetpathnumber(\&pathnumber);
\\\}\\
BTXclose();\\ENclose();\\
return(0);\\
\}
\section*{Backtracking Toolkit Functions}
\subsection*{int BTXopen(char *f1)}
\begin{enumerate}
\item {\it Description}\\Opens the Backtracking Toolkit and associates a hydraulic solution
file with the backtracking modeling
\item {\it Arguments}\\
f1: name of binary EPANET hydraulic file.
\item {\it Returns}\\Returns an error code if there are errors, otherwise returns 0.
\item {\it Notes}\\The binary hydraulic file can be generated in
previous application or in the current application with EPANET
Programmer's Toolkit functions.
\end{enumerate}

\subsection*{int BTXinit(float impactstep, float thresh)}
\begin{enumerate}
\item {\it Description}\\Alllocates memory for data structure and read hydraulic
data from EPANET hydraulic file.
\item {\it Arguments}\\impactstep: time step to integrate impact
coefficients, in the unit of hour.\\thresh: threshold for particle
impact coefficient to delete backtracking particles with small
impact on output water quality.
\item {\it Returns}\\Returns an error code if there are errors, otherwise returns 0.
\end{enumerate}

\subsection*{int BTXsim(char *outputid, float sampletime)}
\begin{enumerate}
\item {\it Description}\\Does backtracking water quality modeling at a node with ID outputid
and at sampletime.
\item {\it Arguments}
\\outputid: ID of the output node.\\sampletime: backtracking modeling time (in
hours).\item {\it Returns}\\Returns an error code if there are
errors, such as invalid output ID, otherwise
returns 0.
\end{enumerate}

\subsection*{int BTXout(float *result)}
\begin{enumerate}
\item {\it Description}\\Calculates water quality modeling at the node and
time specified by \texttt{BTXsim}.
\item {\it Arguments}
\\result: output water quality value (in mg/L).
\item {\it Returns}\\Returns 0.
\item {\it Notes}\\Should be called after BTXsim specifies the
output node and sampling time and the result also depends on the
source strength set by function \texttt{BTXsetinputstrength}.
\end{enumerate}

\subsection*{int BTXsetinput(char * nodeid, int type)}
\begin{enumerate}
\item {\it Description}\\Sets water quality source location or the node whose impact on output water quality is of interest. \item {\it Arguments}
\\nodeid: input node ID.\\type: water quality types that are defined in
\texttt{EPANET2.h}: EN\_CONCEN(0), EN\_MASS(2), EN\_SETPOINT(3) or
EN\_FLOWPACED(4).
\item {\it Returns}\\Returns an error code if there are errors, such as invalid node ID, otherwise returns 0.
\item {\it Notes:}\\The backtracking process stops at any SETPOINT source.
So if the objective is to study the impact of multiple nodes on a
specific output, do not set the source type to be SETPOINT.
\end{enumerate}

\subsection*{int BTXsetinputstrength(char * nodeid, float * strength, int patternlength, float patternstep)}
\begin{enumerate}
\item {\it Description}\\Set water quality source strength.\item {\it Arguments}
\\nodeid: node ID of water quality source.\\strength: vector to store water
quality strength. The index starts with 0.\\patternlength: length of
the vector of strength.\\patternstep: strength pattern step in the
unit of hour.
\item {\it Returns}\\Returns error code if there are errors, such as invalid node ID, otherwise returns 0.
\item {\it Notes:}\\1) The nodeid must be specified as an input using \texttt{BTXsetinput} before water quality source strength can be
assigned.\\2) The source strength vector repeats itself if the
simulation time is longer than the duration that the pattern covers.
\end{enumerate}

\subsection*{int BTXgetimpact(char * nodeid, float * impact, int size)}
\begin{enumerate}
\item {\it Description}\\Gets aggregated impact coefficients of an input node on output water quality
\item {\it Arguments}\\
nodeid: ID of the input node. \\impact: vector to store aggregated
impact coefficients. \\size: length of the vector impact.
\item {\it Returns}\\Returns an error code if there are errors, such as invalid node ID, otherwise returns 0.
\item{\it Notes}\\The vector impact is filled from index 0 to
size-1. The size should be larger than the number of impact
coefficient integration steps, which is the function of output time
and impact coefficient aggregation time step.
\end{enumerate}

\subsection*{int BTXgetpathnumber(int * npath)}
\begin{enumerate}
\item {\it Description}\\Gets number of flow paths from true water
sources and/or SETPOINT water quality sources to output node at
sample time, which are specified by \texttt{BTXsim}.
\item {\it Arguments}\\
npath: integer pointer to store number of flow paths.
\item {\it Returns}\\Returns 0.
\end{enumerate}

\subsection*{int BTXgetpathinfor(int pathindex, int * endnodeindex, float * timedelay, float * impactcoeff)}
\begin{enumerate}
\item {\it Description}\\Gets flow path information.
\item {\it Arguments}\\
pathindex: index of flow path, which should be between 1 and number
of flow paths that can be obtained with
\texttt{BTXgetpathnumber}.\\endnodeindex: integer pointer to store
the index of the node where the backtracked flow path ends.
\\timedelay: the float pointer to store the time delay of the flow
path (in hours).\\impactcoeff: impact coefficient of the flow path.
\item {\it Returns}\\Returns an error code if there are errors, such as invalid path index, otherwise returns 0.
\item{\it Notes:}\\1) The smaller the flow path index, the shorter
the time delay of the path.\\2) Some flow paths end at water sources
that are not water quality sources and the impact coefficients of
such paths are therefore 0.0. Water quality sources have to be
specified using \texttt{BTXsetinput}.
\end{enumerate}

\subsection*{int BTXpipeinpath(char * pipeid, int *flag)}
\begin{enumerate}
\item {\it Description}\\Checks if a pipe is in one or more flow paths.
\item{\it Arguments}\\pipeid: ID of a network pipe.\\flag:
integer pointer to store 1 if true or 0 otherwise.
\item {\it Returns}\\Returns an error code if there are errors, such as invalid pipe index, otherwise returns 0.
\end{enumerate}

\subsection*{int BTXclose()}
\begin{enumerate}
\item {\it Description}\\Frees the memory used by BTX.
\item {\it Returns}\\Returns 0.
\end{enumerate}

\section*{BTX ERROR CODES}
Error 102: no EPANET input file supplied. A standard EPANET input
file was not opened with \texttt{ENopen} before BTX was opened with
\texttt{BTXopen}.\\
\\Error 601: insufficient memory available. There
is not enough physical memory in the computer to do backtracking
analysis.\\
\\Error 602: can not open EPANET hydraulic file. The binary hydraulic
file specified with \texttt{BTXopen} either does not exist or can
not be opened.\\
\\Error 603: hydraulic file error. The binary hydraulic file
specified with \texttt{BTXopen} is not a valid hydraulic file
corresponding to the EPANET input file opened before
\texttt{BTXopen}.\\
\\Error 604: insufficient simulation time. The beginning of the
simulation period is reached before all backtracking particles are
traced back to sources and there for water quality modeled depends
on initial water qualities, which are assumed to be zero for
backtracking analysis.\\
\\Error 605: invalid source type. Invalid EPANET water quality
source types. The source types defined in EPANET header file are
EN\_CONCEN(0), EN\_MASS(1), EN\_SETPOINT(2), and EN\_FLOWPACED(3).\\
\\Error 606: invalid input. A node is not a water quality input. \texttt{BTXsetinputstrength} and \texttt{BTXgetimpact} return this error code if the node has not been specified as an input using \texttt{BTXsetinput}.\\
\\Error 607: invalid node ID.\\
\\Error 608: invalid link ID.\\
\\Error 609: invalid water quality output time. Either less than 0
or larger than the hydraulic simulation duration.\\
\\Error 610: invalid path index. Either less than 1 or larger than
total number of flow paths.
\bibliography{pba}
\end{document}
