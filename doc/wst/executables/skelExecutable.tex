%\subsection{Overview}\label{skelExecutable_skelOverview}
The skeletonizer, \code{spotSkeleton}, reduces the size of a network model by 
grouping neighboring nodes based on the topology of the network and pipe diameter 
threshold. Nodes that are grouped together form a new node, often referred to as a supernode.
The \code{spotSkeleton} executable requires an EPANET 2.00.12 INP network file and a pipe 
diameter threshold. The executable creates a skeletonized EPANET 2.00.12 INP 
network file and map file. The map file defines the nodes that belong to each supernode.

The \code{spotSkeleton} executable includes branch trimming, series pipe merging and 
parallel pipe merging. A pipe diameter threshold determines candidate pipes for 
skeleton steps. The \code{spotSkeleton} executable maintains pipes and nodes with 
hydraulic controls as it creates the skeletonized network. It performs 
series and parallel pipe merges if both pipes are below the pipe diameter 
threshold, calculating hydraulic equivalency for each merge based on the average 
pipe roughness of the joining pipes. For all merge steps, the larger diameter 
pipe is retained. For a series pipe merge, demands (and associated demand 
patterns) are redistributed to the nearest node. Branch trimming removes 
deadend pipes smaller than the pipe diameter threshold and redistributes demands 
(and associated demand patterns) to the remaining junction. The \code{spotSkeleton} 
executable repeats these steps until no further pipes can be removed from the network. 
The \code{spotSkeleton} executable creates an EPANET-compatible skeletonized 2.00.12 network INP file 
and a map file that contains the mapping of original network model nodes into 
skeletonized supernodes. 

Under these skeletonization steps, there is a limit to how much a 
network can be reduced based on its topology, e.g., number of deadend pipes, or 
pipes in series and parallel. For example, sections of the network with a loop, 
or grid, structure will not reduce under these skeleton steps. Additionally, 
the number of hydraulic controls influences skeletonization, as all pipes and 
nodes associated with these features are preserved.

Commercial skeletonization codes include Haestad Skelebrator, MWHSoft 
H2OMAP and MWHSoft InfoWater. 
To validate the \code{spotSkeleton} executable, its output was compared to 
the output of MWHSoft H2OMAP and Infowater. Input parameters were chosen to match \code{spotSkeleton} 
options. Pipe diameter thresholds of 8 inches, 12 inches and 16 inches were 
tested using two large networks. MWHSoft and WST skeletonizers were compared 
using the Jaccard index. The Jaccard index measures similarity 
between two sets by dividing the intersection of the two sets by the union of 
the two sets. In this case, the intersection is the number of pipes that are 
either both removed or not removed by the two skeletonizers, and the union is the 
number of all pipes in the original network. If the two skeletonizers define 
the same supernodes, the Jaccard index equals 1. Skeletonized networks from 
the MWHSoft and WST skeletonizers resulted in a Jaccard index between 0.93 
and 0.95. Thus, the \code{spotSkeleton} executable is believed to be a 
good substitute for commercial skeletonizers.  

\subsection{Usage}\label{skelExecutable_skelUsage}
\begin{unknownListing}
   spotSkeleton <input inp file> <pipe diameter threshold> <output inp file> <output map file>
\end{unknownListing}

\subsection{Arguments}\label{skelExecutable_skelArguments}
\begin{unknownListing}
     <input inp file>
     The input EPANET 2.00.12 INP file to be skeletonized.
     
     <pipe diameter threshold>
     The pipe diameter threshold that determines which pipes might be skeletonized.
     
     <output inp file>
     The output EPANET 2.00.12 INP file created after skeletonization.
     
     <output map file>
     The output map file that contains the mapping of original network nodes to 
     skeletonized supernodes.	 
\end{unknownListing}

\if 0
\subsection{Example}\label{skelExecutable_skelExample}
The EPANET network file Net6.inp contains 
3323 nodes, 3829 pipes, 34 tanks, 1 reservoir, 61 pumps and 2 valves. 
This file is distributed with WST in the examples/networks folder. 
The network is shown in Figure \ref{fig:skelExecutable-origNetwork}. 
\begin{figure}[h]
  \centering
  \includegraphics[scale=0.80]{graphics/Net6.pdf}
  \caption{Original Net6 network.}
  \label{fig:skelExecutable-origNetwork}
\end{figure}

To skeletonize the network based on a 16-inch pipe diameter threshold, the following command can be used:
\begin{unknownListing}
   spotSkeleton Net6.inp 16 Net6_16.inp Net6_16.map
\end{unknownListing}

This produces two files: Net6\_16.inp and Net6\_16.map. The INP file is an EPANET 2.00.12 
compatible network file that has been skeletonized based on the rules described above.
The MAP file contains the name of each supernode in Net6\_16.inp and the original 
nodes in Net6.inp that belong to each supernode.  

The skeletonized network Net6\_16.inp contains 
1128 nodes and 1553 pipes. The number of tanks, reservoir, pumps and valves remains the same. 
The skeletonized network is shown in Figure \ref{fig:skelExecutable-skelNetwork}. 
\begin{figure}[h]
  \centering
  \includegraphics[scale=0.80]{graphics/Net6_16.pdf}
  \caption{Skeletonized Net6 network using a 16-inch pipe diameter threshold.}
  \label{fig:skelExecutable-skelNetwork}
\end{figure}

\subsection{Notes}\label{skelExecutable_picoNotes}
\begin{itemize}
\item Two-tiered sensor placement requires mapping 
between the skeletonized and original network (map file). Commercial skeletonizers 
do not provide this information directly. The MWHSoft Infowater/H2OMAP history 
log file provides it indirectly. The history log file 
tracks sequential skeletonization steps and records merge-\/and-\/trim operations on a 
pipe-by-pipe basis.    
\end{itemize} 
\fi
