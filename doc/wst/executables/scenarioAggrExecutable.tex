%\subsection{Overview}\label{scenarioAggrExecutable_scenarioAggrOverview}
The \code{scenarioAggr} executable takes an impact file and produces an 
aggregated impact file. The \code{scenarioAggr} executable reads an impact file, finds similar 
incidents, combines them and writes out another impact file. The convention is 
to append the string aggr to the output.

The following files are generated during the execution of \code{scenarioAggr}, 
assuming that the input was named network.impact: 
\begin{itemize}
\item aggrnetwork.impact -\/ The new impact file. 
\item aggrnetwork.impact.prob -\/ The probabilities of the aggregated incidents. 
These are non-\/uniform, so any solver must recognize incident probabilities.
\end{itemize}

Not all of the solvers available in the \code{sp} command can perform optimization with 
aggregated impact files. In particular, the heuristic GRASP solver does not 
currently support aggregation because it does not use contamination incident probabilities. 
The Lagrangian and PICO solvers support contamination incident aggregation. However, initial 
results suggest that although the number of contamination incidents is reduced significantly, 
the number of impacts might not be, and solvers might not run much faster. 

\subsection{Usage}\label{scenarioAggrExecutable_scenarioAggrUsage}
\begin{unknownListing}
   scenarioAggr --numEvents=<num_incidents> <impact file>
\end{unknownListing}

\subsection{Options}\label{scenarioAggrExecutable_scenarioAggrOptions}
\begin{unknownListing}
    --numEvents=<number>
    The number of contamination incidents that should be aggregated.
\end{unknownListing}

\subsection{Arguments}\label{scenarioAggrExecutable_scenarioAggrArguments}
\begin{unknownListing}
     <impact-file>
     The input impact file.
\end{unknownListing}

%\subsection{Description}\label{scenarioAggrExecutable_scenarioAggrDescription}
%The \code{scenarioAggr} executable reads an impact file, finds similar 
%incidents, combines them, and writes out another impact file. The convention is 
%to append the string \char`\"{}aggr\char`\"{} to the output.
%
%The following files are generated during the execution of \code{scenarioAggr}, 
%assuming that the input was named \char`\"{}network.impact\char`\"{}: 
%\begin{itemize}
%\item aggrnetwork.impact -\/ the new impact File \ref{formats_impactFile}
%\item aggrnetwork.impact.prob -\/ the probabilities of the aggregated incidents. 
%These are non-\/uniform, so any solver must recognize incident probabilities.
%\end{itemize}
%
%\subsection{Notes}\label{scenarioAggrExecutable_scenarioAggrNotes}
%\begin{itemize}
%\item Not all solvers in SPOT can perform optimization with aggregated impact files. 
%In particular, the heuristic GRASP solver does not currently support aggregation 
%because it does not use incident probabilities. The Lagrangian and PICO solvers 
%support incident aggregation. However, initial results suggest that although the 
%number of incidents is reduced significantly, the number of impacts may not be, 
%and solvers may not run much faster. 
%\end{itemize}
