%\subsection{Overview}\label{evalsensorExecutable_evalsensorOverview}
The \code{evalsensor} executable is used to compute information about the impact 
of contamination incidents for one or more sensor network designs. The \code{evalsensor} 
executable takes a sensor network design in a sensor placement file (see File Formats Section 
\ref{formats_sensorPlacementFile} for more detail) and evaluates them using data from 
an impact file or a list of impact files (see File Formats Section \ref{formats_impactFile}). 
This executable measures the performance of each sensor network designs with respect 
to the set of possible contamination scenarios.

Section \ref{solvers_solvers6} provides more information and an example application of 
this executable.

\subsection{Usage}\label{evalsensorExecutable_evalsensorUsage}
Usage with a specific sensor network design:
\begin{unknownListing}
   evalsensor [options...] <sensor-file> <impact-file1> [<impact-file2>...]
\end{unknownListing}
Usage without a sensor network design:
\begin{unknownListing}
   evalsensor [options...] none <impact-file1> [<impact-file2>...]
If none, is specified, then evalsensor will evaluate impacts without any sensors.
\end{unknownListing}

\subsection{Options}\label{evalsensorExecutable_evalsensorOptions}
%\lstinputlisting{examples/simple/command7.txt}
\begin{unknownListing}
     --all-locs-feasible
     A boolean flag to indicate that all locations are treated as feasible.
     
     --costs=<filename>
     The name of a file that contains the cost information for each node in the network.
	 For more details about the cost file, see File Formats Section `\ref{formats_costFile}`.
     
     --debug    
     A boolean flag to add output information about each incident.

     --format=<type>
     The type of output that the evaluation will generate:
		cout - 	Generates output that is easily read. (default)
		xls - 	Generates output that is easily imported into a MS Excel spreadsheet.
		xml - 	Generates an XML-formatted output to communicate 
				with the TEVA-SPOT GUI. (not currently supported)
		      
     --gamma=<num>
     The fraction of the tail distribution used to compute the VaR and TCE 
	 performance measures. (default is 0.05)
	 
	 -h, --help
	 A boolean flag to display usage information.
	 
	 --incident-weights=<filename>
	 The name of a file that contains the weights of the different contamination incidents. 
	 For more details about the weights file, see File Formats Section `\ref{formats_weightFile}`.
	 
	 --nodemap=<filename>
	 The name of a file that contains the node map information for translating sensor placement 
	 indices to EPANET node IDs. For more details about the nodemap file, see File Formats 
	 Section `\ref{formats_nodeFile}`.
	 
	 -r, --responseTime=<num>
	 The number of minutes that are needed to respond to the detection of 
	 contamination. As the response time increases, the impact increases 
	 because the contaminant affects the network for a greater length of 
	 time.
	 
	 --sc-probabilities=<filename>
	 The name of a file that contains the probability of detection for each sensor category. 
	 For more details about the imperfect sensor class file, see File Formats Section 
	 `\ref{formats_sensorClass}`.
	 
	 --scs=<filename>
	 The name of a file that contains the sensor category information for each possible 
	 sensor location in the network. For more details about the imperfect junction class file, 
	 see File Formats Section `\ref{formats_junctionClass}`.
	 
	 --version
	 A boolean flag to display version information.
	 
	 Note: Options like reponseTime can be specified with the syntax
	 --responseTime 10.0 or --responseTime=10.0.
\end{unknownListing}

\subsection{Arguments}\label{evalsensorExecutable_evalsensorArguments}
\begin{unknownListing}
     <sensor-file>
     A sensor placement file that contains one or more sensor network designs 
	 that will be evaluated. If none, is specified, then evalsensor will evaluate 
	 impacts without any sensors.
     
     <impact-file>
     A impact file that contains the impact data concerning the simulated contamination 
	 incidents. If one or more impact files are specified, then evaluations are 
	 performed for each impact separately.
\end{unknownListing}

%\subsection{Description}\label{evalsensorExecutable_evalsensorDescription}
%The \code{evalsensor} executable takes sensor placements in a Sensor Placement File 
%\ref{formats_sensorPlacementFile} and evaluates them using data from an Impact File 
%\ref{formats_impactFile} (or a list of impact files). This executable measures the 
%performance of each sensor placement with respect to the set of possible 
%contamination locations.
%
%See Section \ref{solvers_solvers6} for further description of this command.

