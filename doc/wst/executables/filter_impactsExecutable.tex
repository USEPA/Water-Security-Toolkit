%\subsection{Overview}\label{filter_impactsExecutable_filter_impactsOverview}
The \code{filter\_\-impacts} script filters out the low-\/impact incidents from 
an impact file. The \code{filter\_\-impacts} command reads an impact file, filters out the 
low-\/impact incidents, rescales the impact values and outputs another 
impact file.

\subsection{Usage}\label{filter_impactsExecutable_filter_impactsUsage}
\begin{unknownListing}
   filter_impacts [options...] <impact-file> <out-file>
\end{unknownListing}

\subsection{Options}\label{filter_impactsExecutable_filter_impactsOptions}
\begin{unknownListing}
     --threshold=<val>
     The contamination incidents with undetected impacts above a specified threshold should be kept.
     
     --percent=<num>
     The percentage of contamination incidents with the worst undetected impact that should be kept.
     
     --num=<num>    
     The number of contamination incidents with the worst undetected impact that should be kept.

     --rescale
     Rescale the impacts using a log10 scale.
     
     --round
     Round input values to the nearest integer.
\end{unknownListing}

\subsection{Arguments}\label{filter_impactsExecutable_filter_impactsArguments}
\begin{unknownListing}
     <impact-file>
     The input impact file.
     
     <out-file>
     The output impact file.
\end{unknownListing}

%\subsection{Description}\label{filter_impactsExecutable_filter_impactsDescription}
%
%The \code{filter\_\-impacts} command reads an impact file, filters out the 
%low-\/impact incidents, rescales the impact values, and writes out another 
%impact file.
%
%\subsection{Notes}\label{filter_impactsExecutable_filter_Notes}
%
%None. 
