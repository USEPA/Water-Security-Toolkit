%\subsection{Overview}\label{measuregenExecutable_measuregenOverview}
The \code{measuregen} executable is used to create a simulated measurements file 
for sensor locations in a water distribution network model. 
A node-time-concentration list is obtained by water quality simulations performed in Merlion.
The MEASURE.dat file contains the node-time-concentration list and can be used to perform source inversion. 

\subsection{Usage}\label{measuregenUsage}
\begin{unknownListing}
   measuregen [options...] <required network option> <required scenario option> <sensor-file> 
\end{unknownListing}

\subsection{Options}\label{measuregenOptions}
\begin{unknownListing}
Data format option:
     --output-prefix=<filename>         
     The name to add to all output files.
 
EPANET input file options:
     --quality-timestep-minutes=<num>
     The size of the water quality time step used by Merlion to perform the water quality simulations. 
     When this value is specified, it overrides the value in the EPANET 2.00.12 input file.
     
     --simulation-duration-minutes=<num>
     The length of water quality simulation used to build the Merlion water quality model. 
     When this value is specified, it overrides the value in the EPANET 2.00.12 input file. This is useful 
     when the length of the simulation specified in the EPANET 2.00.12 input file is longer than the time 
     horizon in which the sensor measurements are needed. For instance, if the simulation duration 
     in the EPANET 2.00.12 input is seven days, but sensor measurements are only needed for the first three 
     days of the simulation. This option reduces the memory required to build the Merlion water 
     quality model.
 
Label options:
     --custom-label-map=<filename>      
     The name of a file which maps EPANET node names to custom labels. All data files will be written 
     using these custom labels.
     
     --output-merlion-labels 
     Node names will be translated into integer node IDs to reduce file sizes for large networks. 
     A node map is provided to map node IDs back to node names (MERLION_LABEL_MAP.txt). This option
     is overridden by the --custom-label-map flag.
 
Noise options:
     --FNR=<num>                   
     The false negative rate to apply to all sensors.
     
     --FPR=<num>                    
     The false positive rate to apply to all sensors.
     
     --scale=<num>                  
     The scaling value to add noise to the base demand. 
     
     --seed=<num>                   
     The seed to generate the random number used at the moment of adding noise. 
 
Other options:
     --disable-warnings      
     A boolean flag to disables printing of warning statements to stderr.
     
     --enable-logging        
     A boolean flag to generate a log file with verbose runtime information.
	 
     -h, --help                  
     A boolean flag to display usage information.
	 
     --ignore-merlion-warnings
     A boolean flag to ignore warnings about unsupported features of Merlion.
     
     -v, --version               
     A boolean flag to display version information.
 
Save options:
     --epanet-rpt-file       
     A boolean flag to save output file generated by EPANET 2.00.12 during hydraulic simulations.
     
     --merlion-save-file     
     A boolean flag to save the text file defining the Merlion water quality model.
 
Time options:
     --concentrations        
     The concentration values will be printed in the measurement file.
     
     --decay-const=<num>           
     The value for the first-order decay coefficient(1/min). The default value is taken from EPANET 2.00.12  
     input file.
     
     --measures-per-hour=<num>     
     The number of measurements to take per hour. The default value is 60.
     
     --start-sensing-time=<num>    
     The time to start taking measurements. This value is in minutes.
     
     --stop-sensing-time=<num>     
     The time to stop taking measurements. This value is in minutes.
     
     --threshold=<num>             
     The value of concentrations above which the measurements are positive. The default value is 0.0.
 

\end{unknownListing}

\subsection{Arguments}\label{measuregenArguments}
\begin{unknownListing}
     <required network option>
     --inp=<filename>                   
     The name of the EPANET 2.00.12 network file.
     
     --wqm=<filename>                   
     The name of the Merlion water quality model (wqm) file.
 
     <required scenario option>
     This argument defines the injection incidents to simulate in order 
     to obtain measurements at the sensor locations. Three options are 
     available to define the injection incidents.
     
     --scn=<filename>                   
     The name of the SCN file for specifying the injection incidents.
     
     --tsg=<filename>                   
     The name of the TSG file for specifying the injection incidents.
	
     --tsi=<filename>                   
     The name of the TSI file for specifying the injection incidents.
	 
     --tsi-species-id=<name>        
     (*optional) The single TSI species id to use in each scenario by Merlion. All other species 
     will be ignored. If this option is not used and multiple species ids are in the TSI
     file, an error will occur.
     
     <sensor node file>
     A file with a list of sensor node names.
\end{unknownListing}
