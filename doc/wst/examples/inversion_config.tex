\begin{description}[topsep=0pt,parsep=0.5em,itemsep=-0.4em]
  \item[{network}]\hfill
  \begin{description}[topsep=0pt,parsep=0.5em,itemsep=-0.4em]
    \item[{epanet file}]\hfill
\\ The name of the EPANET 2.00.12 input (INP) file that defines the water distribution
                network model.
                
                Required input.
  \end{description}
  \item[{measurements}]\hfill
  \begin{description}[topsep=0pt,parsep=0.5em,itemsep=-0.4em]
    \item[{grab samples}]\hfill
\\The name of the file that contains all the measurements from 
                the manual grab samples and the fixed sensors. The measurement file 
                format is documented in File Formats Section \ref{formats_measFile}.

                Required input.
  \end{description}
  \item[{inversion}]\hfill
  \begin{description}[topsep=0pt,parsep=0.5em,itemsep=-0.4em]
    \item[{algorithm}]\hfill
\\The algorithm used to perform source inversion. The options are 
				optimization, bayesian, or csa. The optimization algorithm requires 
				AMPL or PYOMO along with a MIP solver. The bayesian algorithm 
				uses Bayes' Rule to update probability of a particular node 
				being the contaminant source node. The CSA is the Contaminant
                                Status Algorithm by \citet{csa}.
                
                Required input, default = optimization.
    \item[{formulation}]\hfill
\\The formulation used by the optimization algorithm. The options are 
                LP\_discrete (discrete LP), MIP\_discrete (discrete MIP), 
                MIP\_discrete\_nd (discrete MIP with no decrease) or 
                MIP\_discrete\_step (discrete MIP for step injection).
                
                Required input for optimization algorithm, default = MIP\_discrete.
    \item[{model format}]\hfill
\\The modeling language used to build the formulation specified
                by the formulation option. The options are AMPL and PYOMO. 
				AMPL is a third party package that must be installed by 
				the user if this option is specified. PYOMO is an open source 
				software package that is distributed with WST.
                
                Required input for optimization algorithm, default = PYOMO.
    \item[{merlion water quality model}]\hfill
\\This option is set to true to use the Merlion 
                water quality model for simulating the candidate injections
                in the Bayesian probability-based method. It can be set to false
                to use EPANET 2.00.12 for these simulations. Note that the Merlion water quality
                model is required in either case to generate the initial list of candidate injections.
 
                Optional input, default = true.
    \item[{horizon}]\hfill
\\The minutes over which the past measurement
                data is used for source inversion. It is calculated backwards from
                the latest measurement time in the measurements file. 
                All measurements in the measurements file that are within the horizon 
                are used (both negative and positive). In the case of the CSA algorithm, 
                the implementation assumes fixed sensors only, and all measurements at 
                these sensors are assumed to be negative prior to the horizon.
                If the horizon is longer
                than the time between the latest measurement and simulation start time,
                then all the measurements are used for source inversion.
                
                Required input, default = None (Start of simulation).
    \item[{num injections}]\hfill
\\The number of possible injections to consider when
                performing inversion. Multiple injections are only supported by
                the MIP formulation. This value must be set to 1 for the LP model
                or the probability algorithm.
                
                Required input for optimization algorithm, default = 1.
    \item[{measurement failure}]\hfill
\\The probability that a sensors gives an incorrect reading. Must be between 0 and 1. 
                
                Required input for the Bayesian algorithm, default = 0.05.
    \item[{positive threshold}]\hfill
\\The concentration threshold used by the sensors to flag a positive 
                detection measurement. This is a parameter in the optimization algorithm (Equation \ref{eqn.milp_positive_set}).
                
                Required input for optimization algorithm, default = 100 mg/L. 
    \item[{negative threshold}]\hfill
\\The concentration threshold used by the sensors to flag
                a negative detection measurement. This is a parameter in the
                optimization algorithm (Equation \ref{eqn.milp_negative_set}).
                
                Required input for optimization algorithm, default = 0.0 mg/L.
    \item[{feasible nodes}]\hfill
\\A list that defines nodes that can be considered for the source inversion problem.
                The options are: (1) ALL, which specifies all nodes as feasible source locations;
                (2) NZD, which specifies all non-zero demand nodes as feasible source locations;
                (3) a list of EPANET node IDs, which identifies specific nodes as feasible source locations; or
                (4) a filename, which is a space or comma separated file containing a list of 
                specific nodes as feasible source locations. 

                Optional input.
    \item[{candidate threshold}]\hfill
\\The objective cut-off value for candidate contamination incidents 
                using the optimization algorithm. The objective value represents the
                likelihood of a particular node being the injection node (See Equation \ref{eq_transform_obj}).  
                The objective values are normalized to 1 and only the nodes having 
                their objective values greater or equal to the threshold are reported
                in the inversion results. 
                
                Required input for optimization algorithm, default = 0.20.
    \item[{confidence}]\hfill
\\The probability cut-off value for candidate contamination incidents
                using the Bayesian algorithm. The value is between 0 and 1. 
                
                Required for the Bayesian algorithm, default = 0.95.
    \item[{output impact nodes}]\hfill
\\A option to output a Likely\_Nodes.dat file that contains only
                the node IDs of the possible contaminant injection nodes obtained from the 
                \code{inversion} subcommand. This file can be used as the feasible nodes for the next 
                iteration of the \code{inversion} subcommand to only consider this set of possible contaminant 
                injection nodes.
                
                Optional input, default = false.
  \end{description}
  \item[{solver}]\hfill
  \begin{description}[topsep=0pt,parsep=0.5em,itemsep=-0.4em]
    \item[{type}]\hfill
\\The solver type. Each component of WST
				(e.g., sensor placement, flushing response, booster 
				placement) has different 
				solvers available. More specific details are provided in 
				the subcommand's chapter.
                
                Required input.
    \item[{options}]\hfill
\\A list of options associated with a specific solver type. More
            information on the options available for a specific solver
            is provided in the solver's documentation. The Getting
            Started Section \ref{dependencies} provides links to the
            different solvers.
            
            Optional input.
    \item[{threads}]\hfill
\\The maximum number of threads or function evaluations the solver is
                allowed to use.  This option is not available to all solvers or all analyses.
                
                Optional input.
    \item[{logfile}]\hfill
\\The name of a file to output the results of the solver.
                
                Optional input.
    \item[{verbose}]\hfill
\\The solver verbosity level.
                
                Optional input, default = 0 (lowest level).
    \item[{initial points}]\hfill
    \begin{description}[topsep=0pt,parsep=0.5em,itemsep=-0.4em]
      \item[{nodes}]\hfill
\\A list of node locations (EPANET IDs) to begin the optimization
        process. Currently, this option is only supported for the
        network solver used in the flushing and booster\_msx
        subcommands. This input causes an error for other subcommands.
        
        Optional input.
      \item[{pipes}]\hfill
\\A list of pipe locations (EPANET IDs) to begin the optimization
        process. Currently, this option is only supported for the
        network solver used in the flushing subcommand. This input causes an error for other subcommands.
        
        Optional input.
    \end{description}
  \end{description}
  \item[{configure}]\hfill
  \begin{description}[topsep=0pt,parsep=0.5em,itemsep=-0.4em]
    \item[{output prefix}]\hfill
\\The prefix used for all output files.
                
                Required input.
    \item[{output directory}]\hfill
      \\The output directory to store the results.
    \item[{debug}]\hfill
\\The debugging level (0 or 1) that indicates the amount of debugging 
                information printed to the screen, log file and output yml file. 
                
                Optional input, default = 0 (lowest level).
  \end{description}
\end{description}
