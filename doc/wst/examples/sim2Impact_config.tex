\begin{description}[topsep=0pt,parsep=0.5em,itemsep=-0.4em]
  \item[{impact}]\hfill
  \begin{description}[topsep=0pt,parsep=0.5em,itemsep=-0.4em]
    \item[{erd file}]\hfill
\\The name of the ERD database file that contains the 
                contaminant transport simulation results. It is 
                created by running the \code{tevasim} subcommand.
                Multiple ERD files (entered as a list, i.e., [<file1>, <file2>]) can be combined to
                generate a single impact file. This can be used to combine
                simulation results from different types of contaminants, in
                which the ERD files were generated from different
                TSG files.
                
                Required input.
    \item[{metric}]\hfill
\\The impact metric used to compute the impact file. Options
                include EC, MC, NFD, PD, PE, PK, TD or VC. One impact file 
                is created for each metric selected. These metrics are 
                defined in Section \ref{impact_measures}.
                
                Required input.
    \item[{tai file}]\hfill
\\The name of the TAI file that contains health impact information. 
                The TAI file format is documented in File Formats Section \ref{formats_taiFile}.
                
                Required input if a public health metric is used (PD, PE or PK).
    \item[{response time}]\hfill
\\The number of minutes that are needed to respond to the
                detection of a contaminant. This represents the time that it takes
                a water utility to stop the spread of the contaminant in the network and 
                eliminate the consumption of contaminated water. As the response time increases,
                the impact increases because the contaminant affects the network
                for a greater length of time.  
                
                Required input, default = 0 minutes.
    \item[{detection limit}]\hfill
\\The minimum concentration that must be exceeded before a sensor can detect a contaminant.
                There must be one threshold for each ERD file. The units of
                these detection limits depend on the units of the contaminant
                simulated for each ERD file (e.g., number of cells of a
                biological agent).  
                
                Required input, default = 0.
    \item[{detection confidence}]\hfill
\\The number of sensors that must detect an incident before
                the impacts are calculated.  
                
                Required input, default = 1 sensor.
  \end{description}
  \item[{configure}]\hfill
  \begin{description}[topsep=0pt,parsep=0.5em,itemsep=-0.4em]
    \item[{output prefix}]\hfill
\\The prefix used for all output files.
                
                Required input.
    \item[{output directory}]\hfill
      \\The output directory to store the results.
    \item[{debug}]\hfill
\\The debugging level (0 or 1) that indicates the amount of debugging 
                information printed to the screen, log file and output yml file. 
                
                Optional input, default = 0 (lowest level).
  \end{description}
\end{description}
