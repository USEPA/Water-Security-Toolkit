\begin{description}[topsep=0pt,parsep=0.5em,itemsep=-0.4em]
  \item[{impact data}]\hfill
  \begin{description}[topsep=0pt,parsep=0.5em,itemsep=-0.4em]
    \item[{name}]\hfill
\\The name of the impact block that is used in the objective or constraint block.
                
                Required input.
                
    \item[{impact file}]\hfill
\\The name of the impact file that is created by \code{sim2Impact} and 
                contains the detection time and the total
                impact given a sensor at that node is the first to detect
                contamination from that scenario. 
                The impact file format is documented in File Formats Section \ref{formats_impactFile}.
                
                Required input.
                
    \item[{nodemap file}]\hfill
\\The name of the nodemap file that is created by \code{sim2Impact} and 
                maps sensor placement ids to the network node labels. 
                The nodemap file format is documented in File Formats Section \ref{formats_nodeFile}.
                
                Required input.
                
    \item[{weight file}]\hfill
\\The name of the weight file that specifies the weights for contamination
                incidents. This file supports the optimization of weighted
                impact metrics. 
                The weight file format is documented in File Formats Section \ref{formats_weightFile}.
                
                Optional input; by default, incidents are optimized with weight 1.
  \end{description}

  \item[{cost}]\hfill
  \begin{description}[topsep=0pt,parsep=0.5em,itemsep=-0.4em]
    \item[{name}]\hfill
\\The name of the cost block that is used in the objective or constraint block.
                
                Optional input.
    \item[{cost file}]\hfill
\\The name of the cost file that contains the costs for the installation of
                sensors throughout the distribution network. This file contains
                EPANET ID/cost pairs.
                The cost file format is documented in File Formats Section \ref{formats_costFile}.
                
                Optional input.
  \end{description}
  \item[{objective}]\hfill
  \begin{description}[topsep=0pt,parsep=0.5em,itemsep=-0.4em]
    \item[{name}]\hfill
\\The name of the objective block that is used in sensor placement block.
                
                Required input.
    \item[{goal}]\hfill
\\The objective of the optimization process that defines what is going to minimized. 
                The options are the name of the impact block, the name of the cost block, 
                the number of sensors (NS) or the number of failed detections (NFD).
				
				Required input.
    \item[{statistic}]\hfill
      \\The objective statistic. The TOTAL
      statistic is used when the goal is NS                             or
      NFD. When the goal is to compute a                             statistic
      of an impact block, the options                             are MEAN,
      MEDIAN, VAR, TCE, CVAR, TOTAL                             or WORST. For
      example, MEAN will minimize                             the mean impacts
      over all of the                             contamination scenarios,
      while WORST                             will only minimize the worst
      impacts                             from the ensemble of contamination
      scenarios.                                  Required input.
    \item[{gamma}]\hfill
      \\The value of gamma that specifies the                 fraction of the
      distribution of impacts that will                 be used to compute the
      VAR, CVAR and TCE statistics.                 Gamma is assumed to be in
      the interval (0,1], which means                  that gamma can be
      greater than zero but less than or equal to one. It                 can
      be interpreted as specifying the 100*gamma                 percent of
      the worst contamination incidents that                 are used for
      these calculations.                  Required input for VAR or CVAR
      objective statistics,                 default = 0.05.
  \end{description}
  \item[{constraint}]\hfill
  \begin{description}[topsep=0pt,parsep=0.5em,itemsep=-0.4em]
    \item[{name}]\hfill
\\The name of the constraint block that is used in sensor placement block.
                
                Required input.
    \item[{goal}]\hfill
\\The constraint goal. The options are the name of the impact block name, 
                the name of the cost block, the number of sensors (NS) or the number of failed detections (NFD).
                
                Required input.
    \item[{statistic}]\hfill
\\The constraint statistic. The TOTAL
                statistic is used when the goal is NS
                or NFD. When the goal is to compute a
                statistic of an impact block, the options
                are MEAN, MEDIAN, VAR, TCE, CVAR, TOTAL
                or WORST. For example, MEAN will constrain
                the mean impacts over all of the
                contamination scenarios, while WORST
                will only constrain the worst impacts
                from the ensemble of contamination
                scenarios.

                Required input.
    \item[{gamma}]\hfill
\\The value of gamma that specifies the fraction of the distribution of impacts that
                will be used to compute the VAR, CVAR and TCE statistics. Gamma
                is assumed to be in the interval (0,1], which means that gamma 
				can be greater than zero but less than or equal to one. It can be interpreted
                as specifying the 100*gamma percent of the worst contamination
                incidents that are used for these calculations.  
                
                Required input for VAR or CVAR objective statistics, default = 0.05.
    \item[{bound}]\hfill
\\The upper bound on the constraint.
                
                Optional input.
  \end{description}
  \item[{aggregate}]\hfill
  \begin{description}[topsep=0pt,parsep=0.5em,itemsep=-0.4em]
    \item[{name}]\hfill
\\The name of the aggregation block that is used in sensor placement block.
                
                Optional input.
    \item[{type}]\hfill
\\The type of aggregation used to reduce the size of the sensor placement problem.  
                The options are THRESHOLD, PERCENT or RATIO.

                THRESHOLD is used to aggregate similar impacts by specifying a goal and a value.  
                This is used to reduce the total size of the sensor placement formulation (for large problems).
                Solutions generated with non-zero thresholds are not guaranteed
                to be globally optimal.

                PERCENT is an alternative method to compute the aggregation threshold 
                in which the value (of the goal-value pair) is a real number between 0.0 and 1.0. 
                Over all contamination incidents, compute the maximum difference, d, between the impact of the
                contamination incident if it is not detected and the impact if it is detected
                at the earliest possible feasible location and set the aggregation threshold to 
                d times the aggregation percent. If both THRESHOLD and PERCENT are set to valid values, 
                then PERCENT takes priority.

                RATIO is also specified with a goal-value pair in which value is a real number between
                0.0 and 1.0.
                
                Optional input.
    \item[{goal}]\hfill
\\The aggregation goal for the aggregation type.
                
                Optional input.
    \item[{value}]\hfill
\\The aggregation value for the aggregation type. If the aggregation type is PERCENT or RATIO,
                then this value is a real number between 0.0 and 1.0.
                
                Optional input.
    \item[{conserve memory}]\hfill
\\The maximum number of impact files that should be read into memory 
                at any one time. This option allows impact files to be processed in 
                a memory conserving mode if location aggregation is chosen and the original impact
                files are very large. For example, a conserve memory value of 10000 requests
                that no more than 10000 impacts should be read into memory at any one 
                time while the original impact files are being processed into smaller
                aggregated files.  
				
				Optional input, default = zero to turn off this option.
    \item[{distinguish detection}]\hfill
\\A goal for which aggregation should not allow incidents to
                become trivial. If the aggregation threshold is so large that all
                locations, including the dummy, would form a single superlocation,
                this forces the dummy to be in a superlocation by itself. Thus,
                the sensor placement will distinguish between detecting and not
                detecting. This option can be listed multiple times, to specify
                multiple goals. 
                
                Optional input, default = 0.
    \item[{disable aggregation}]\hfill
\\Disable aggregation for this goal, even at value zero, which
                would incur no error. Each witness incident will be in a separate
                superlocation. This option can be listed multiple times to
                specify multiple goals. ALL can be used to specify
                all goals. 
                
                Optional input, default = 0.
  \end{description}
  \item[{imperfect}]\hfill
  \begin{description}[topsep=0pt,parsep=0.5em,itemsep=-0.4em]
    \item[{name}]\hfill
\\The name of the imperfect block that is used in sensor placement block.
                
                Optional input.
    \item[{sensor class file}]\hfill
\\The name of the imperfect sensor class file that defines the detection probabilities
                for all sensor categories. It is used with the imperfect-sensor model 
                and must be specified in conjunction with a imperfect junction class file.
                The imperfect sensor class file format is documented in File Formats Section \ref{formats_sensorClass}.
                
                Optional input.
    \item[{junction class file}]\hfill
\\The name of the imperfect junction class file that defines a sensor category for
                each network node. It is used with the imperfect-sensor model and
                must be specified in conjunction with a imperfect sensor class file.
                The imperfect junction class file format is documented in File Formats Section \ref{formats_junctionClass}.
                
                Optional input.
  \end{description}
  \item[{sensor placement}]\hfill
  \begin{description}[topsep=0pt,parsep=0.5em,itemsep=-0.4em]
    \item[{type}]\hfill
\\The sensor placement problem type. The command \code{wst sp ---help-problems} 
                provides a list of problem types for sensor placement. For example, average-case perfect-sensor
                is the standard problem type for sensor placement, since it uses the mean statistic, 
                zero constraints, single objective and perfect sensors. 
                
                Required option, default = average-case perfect-sensor.
    \item[{modeling language}]\hfill
\\The modeling language to generate the sensor placement optimization 
                problem. The options are NONE, PYOMO or AMPL. 

                Required input, default = NONE.
    \item[{objective}]\hfill
\\The name of the objective block previously defined to be used in sensor placement.
                
                Required input.
    \item[{constraint}]\hfill
\\The name of the constraint block previously defined to be used in sensor placement.
                
                Required input.
    \item[{imperfect}]\hfill
\\The name of the imperfect block previously defined to be used in sensor placement.
                
                Optional input.
    \item[{aggregate}]\hfill
\\The name of the aggregate block previously defined to be used in sensor placement.
                
                Optional input.
    \item[{compute bound}]\hfill
\\A flag to indicate if bounds should be computed on the sensor placement
                solution. The options are true or false. 
                
                Optional input, default = false.
    \item[{presolve}]\hfill
\\A flag to indicate if the sensor placement problem should be presolved. 
                The options are true or false. 
                
                Optional input, default = true.
    \item[{compute greedy ranking}]\hfill
\\A flag to indicate if a greedy ranking of the sensor locations should be calculated. 
                The options are true or false. 
                
                Optional input, default = false.
    \item[{location}]\hfill
    \begin{description}[topsep=0pt,parsep=0.5em,itemsep=-0.4em]
      \item[{feasible nodes}]\hfill
\\A list that defines nodes that can be considered for the sensor placement problem.
                The options are:
                (1) ALL, which specifies all nodes as feasible sensor locations;
                (2) NZD, which specifies all non-zero demand nodes as feasible sensor locations;
                (3) a list of EPANET node IDs, which identifies specific nodes as feasible sensor locations; 
                (4) a filename, which references a space or comma separated file containing a list of 
                specific nodes as feasible sensor locations; or
                (5) NONE, which indicates this option is ignored.
                
                Required input, default = ALL.
      \item[{infeasible nodes}]\hfill
\\A list that defines nodes that cannot be considered for the sensor placement problem.
                The options are:
                (1) ALL, which specifies all nodes as infeasible sensor locations;
                (2) NZD, which specifies non-zero demand nodes as infeasible sensor locations;
                (3) a list of EPANET node IDs, which identifies specific nodes as infeasible sensor locations;
                (4) a filename, which references a space or comma separated file containing a list of 
                specific nodes as infeasible sensor locations; or
                (5) NONE, which indicates this option is ignored.
                
                Optional input, default = NONE.
      \item[{fixed nodes}]\hfill
\\A list that defines nodes that are already sensor locations.
                The options are:
                (1) ALL, which specifies all nodes as fixed sensor locations;
                (2) NZD, which specifies non-zero demand nodes as fixed sensor locations;
                (3) a list of EPANET node IDs, which identifies specific nodes as fixed sensor locations; 
                (4) a filename, which references a space or comma separated file containing a list of 
                specific nodes as fixed sensor locations; or
                (5) NONE, which indicates this option is ignored.
                
                Optional input, default = NONE.
      \item[{unfixed nodes}]\hfill
\\A list that defines nodes that are unfixed sensor locations.
                The options are:
                (1) ALL, which specifies all nodes as unfixed sensor locations;
                (2) NZD, which specifies non-zero demand nodes as unfixed sensor locations;
                (3) a list of EPANET node IDs, which identifies specific nodes as unfixed sensor locations; 
                (4) a filename, which references a space or comma separated file containing a list of 
                specific nodes as unfixed sensor locations; or
                (5) NONE, which indicates this option is ignored.
                
                Optional input, default = NONE.
    \end{description}
  \end{description}
  \item[{solver}]\hfill
  \begin{description}[topsep=0pt,parsep=0.5em,itemsep=-0.4em]
    \item[{type}]\hfill
\\The solver type. Each component of WST
				(e.g., sensor placement, flushing response, booster 
				placement) has different 
				solvers available. More specific details are provided in 
				the subcommand's chapter.
                
                Required input.
    \item[{options}]\hfill
\\A list of options associated with a specific solver type. More
            information on the options available for a specific solver
            is provided in the solver's documentation. The Getting
            Started Section \ref{dependencies} provides links to the
            different solvers.
            
            Optional input.
    \item[{threads}]\hfill
\\The maximum number of threads or function evaluations the solver is
                allowed to use.  This option is not available to all solvers or all analyses.
                
                Optional input.
    \item[{logfile}]\hfill
\\The name of a file to output the results of the solver.
                
                Optional input.
    \item[{verbose}]\hfill
\\The solver verbosity level.
                
                Optional input, default = 0 (lowest level).
    \item[{initial points}]\hfill
    \begin{description}[topsep=0pt,parsep=0.5em,itemsep=-0.4em]
      \item[{nodes}]\hfill
\\A list of node locations (EPANET IDs) to begin the optimization
        process. Currently, this option is only supported for the
        network solver used in the flushing and booster\_msx
        subcommands. This input causes an error for other subcommands.
        
        Optional input.
      \item[{pipes}]\hfill
\\A list of pipe locations (EPANET IDs) to begin the optimization
        process. Currently, this option is only supported for the
        network solver used in the flushing subcommand. This input causes an error for other subcommands.
        
        Optional input.
    \end{description}
  \end{description}
  \item[{configure}]\hfill
  \begin{description}[topsep=0pt,parsep=0.5em,itemsep=-0.4em]
    \item[{output prefix}]\hfill
\\The prefix used for all output files.
                
                Required input.
    \item[{output directory}]\hfill
      \\The output directory to store the results.
    \item[{debug}]\hfill
\\The debugging level (0 or 1) that indicates the amount of debugging 
                information printed to the screen, log file and output yml file. 
                
                Optional input, default = 0 (lowest level).
  \end{description}
\end{description}
