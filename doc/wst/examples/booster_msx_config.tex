\begin{description}[topsep=0pt,parsep=0.5em,itemsep=-0.4em]
  \item[{network}]\hfill
  \begin{description}[topsep=0pt,parsep=0.5em,itemsep=-0.4em]
    \item[{epanet file}]\hfill
\\ The name of the EPANET 2.00.12 input (INP) file that defines the water distribution
                network model.
                
                Required input.
  \end{description}
  \item[{scenario}]\hfill
  \begin{description}[topsep=0pt,parsep=0.5em,itemsep=-0.4em]
    \item[{location}]\hfill
\\A list that describes the injection locations for the contamination scenarios.
                The options are: (1) ALL, which denotes all nodes (excluding tanks and reservoirs)
                as contamination injection locations; (2) NZD, which denotes all nodes with
                non-zero demands as contamination injection locations; or (3) an EPANET node ID, 
                which identifies a node as the contamination injection location. This allows 
                for an easy specification of single or multiple contamination scenarios.
                
                Required input unless a TSG or TSI file is specified.
    \item[{type}]\hfill
\\The injection type for the contamination scenarios. The options are MASS, CONCEN, FLOWPACED or SETPOINT. 
                See the EPANET 2.00.12 user manual for additional information about source types \citep{EPANETusermanual}.
                
                Required input unless a TSG or TSI file is specified.
    \item[{strength}]\hfill
\\The amount of contaminant injected into the network for the contamination scenarios.  
                If the type option is MASS, then the units for the strength are in mg/min. 
                If the type option is CONCEN, FLOWPACED or SETPOINT, then units are in mg/L.
                
                Required input unless a TSG or TSI file is specified.
    \item[{species}]\hfill
\\The name of the contaminant species injected into the network. This is the name of a single species. 
                It is required when using EPANET-MSX, since multiple species might be simulated, but
                only one is injected into the network. For cases where multiple contaminants are injected,
                a TSI file must be used.
                
                Required input for EPANET-MSX unless a TSG or TSI file is specified.
    \item[{start time}]\hfill
\\The injection start time that defines when the contaminant injection begins. 
                The time is given in minutes and is measured from the start of the simulation. 
                For example, a value of 60 represents an injection that starts at hour 1 of the simulation.
                
                Required input unless a TSG or TSI file is specified.
    \item[{end time}]\hfill
\\The injection end time that defines when the contaminant injection stops.				
                The time is given in minutes and is measured from the start of the simulation.
                For example, a value of 120 represents an injection that ends at hour 2 of the simulation.
                
                Required input unless a TSG or TSI file is specified.
    \item[{tsg file}]\hfill
\\The name of the TSG scenario file that defines the ensemble of contamination
                scenarios to be simulated. Specifying a TSG file will
                override the location, type, strength, species, start and end times options specified in
                the WST configuration file. The TSG file format is documented in File Formats Section \ref{formats_tsgFile}.
                
                Optional input.
    \item[{tsi file}]\hfill
\\The name of the TSI scenario file that defines the ensemble of contamination
                scenarios to be simulated. Specifying a TSI file will
                override the TSG file, as well as the location, type, strength, species, start and end time options specified in
                the WST configuration file. The TSI file format is documented in File Formats Section \ref{formats_tsiFile}.
                
                Optional input.
    \item[{signals}]\hfill
\\Name of file or directory with information to generate 
                or load signals. If a file is provided the list of inp tsg tuples
                 will be simulated and the information stored in signals files. If
                a directory with the signals files is specified, the signal files will
                be read and loaded in memory. This input is only valid for the uq
                subcommand and the grabsample subcommand with probability based formulations.

                Optional input.
    \item[{msx file}]\hfill
\\The name of the EPANET-MSX multi-species file that defines the multi-species reactions to
                be simulated using EPANET-MSX.
                
                Required input for EPANET-MSX.
    \item[{msx species}]\hfill
\\The name of the MSX species whose concentration profile will be saved by the EPANET-MSX simulation
                and used for later calculations.
                
                Required input for EPANET-MSX.
    \item[{merlion}]\hfill
\\A flag to indicate if the Merlion water quality
                simulator should be used. The options are true or false. 
                If an MSX file is provided, EPANET-MSX will be used.
                
                Required input, default = false.
  \end{description}
  \item[{impact}]\hfill
  \begin{description}[topsep=0pt,parsep=0.5em,itemsep=-0.4em]
    \item[{erd file}]\hfill
\\The name of the ERD database file that contains the 
                contaminant transport simulation results. It is 
                created by running the \code{tevasim} subcommand.
                Multiple ERD files (entered as a list, i.e., [<file1>, <file2>]) can be combined to
                generate a single impact file. This can be used to combine
                simulation results from different types of contaminants, in
                which the ERD files were generated from different
                TSG files.
                
                Required input.
    \item[{metric}]\hfill
\\The impact metric used to compute the impact file. Options
                include EC, MC, NFD, PD, PE, PK, TD or VC. One impact file 
                is created for each metric selected. These metrics are 
                defined in Section \ref{impact_measures}.
                
                Required input.
    \item[{tai file}]\hfill
\\The name of the TAI file that contains health impact information. 
                The TAI file format is documented in File Formats Section \ref{formats_taiFile}.
                
                Required input if a public health metric is used (PD, PE or PK).
    \item[{response time}]\hfill
\\The number of minutes that are needed to respond to the
                detection of a contaminant. This represents the time that it takes
                a water utility to stop the spread of the contaminant in the network and 
                eliminate the consumption of contaminated water. As the response time increases,
                the impact increases because the contaminant affects the network
                for a greater length of time.  
                
                Required input, default = 0 minutes.
    \item[{detection limit}]\hfill
\\The minimum concentration that must be exceeded before a sensor can detect a contaminant.
                There must be one threshold for each ERD file. The units of
                these detection limits depend on the units of the contaminant
                simulated for each ERD file (e.g., number of cells of a
                biological agent).  
                
                Required input, default = 0.
    \item[{detection confidence}]\hfill
\\The number of sensors that must detect an incident before
                the impacts are calculated.  
                
                Required input, default = 1 sensor.
  \end{description}
  \item[{booster msx}]\hfill
  \begin{description}[topsep=0pt,parsep=0.5em,itemsep=-0.4em]
    \item[{detection}]\hfill
\\The sensor network design used to detect contamination scenarios. The
                sensor locations are used to compute a detection time for each 
                contamination scenario in the TSG file. The options are a list of 
                EPANET node IDs or a file name which contains a list of EPANET node IDs.
                
                Required input.
    \item[{toxin species}]\hfill
\\The name of the contaminant species that is injected in each
                contamination scenario. This is the species that interacts with the 
                injected disinfectant and whose impact is going to be minimized.
                
                Required input.
    \item[{decon species}]\hfill
\\The name of the decontaminant or disinfectant species that is injected from the 
                booster stations.
                
                Required input.
    \item[{feasible nodes}]\hfill
\\A list that defines nodes that can be considered for the booster station placement problem.
                The options are: (1) ALL, which specifies all nodes as feasible booster station locations;
                (2) NZD, which specifies all non-zero demand nodes as feasible booster station locations;
                (3) NONE, which specifies no nodes as feasible booster station locations;
                (4) a list of EPANET node IDs, which identifies specific nodes as feasible booster station locations; or
                (5) a filename, which references a space or comma separated file containing a list of 
                specific nodes as feasible booster station locations.
                
                Required input, default = ALL.
    \item[{infeasible nodes}]\hfill
\\A list that defines nodes that cannot be considered for the booster station placement problem.
                The options are: (1) ALL, which specifies all nodes as infeasible booster station locations;
                (2) NZD, which specifies non-zero demand nodes as infeasible booster station locations;
                (3) NONE, which specifies no nodes as infeasible booster station locations;
                (4) a list of EPANET node IDs, which identifies specific nodes as infeasible booster station locations; or
                (5) a filename, which references a space or comma separated file containing a list of 
                specific nodes as infeasible booster station locations. 
                
                Optional input, default = NONE.
    \item[{max boosters}]\hfill
\\The maximum number of booster stations that can be placed in the
                network. The value must be a nonnegative integer or a list of
                nonnegative integers. When a list is specified, the optimization
                will be performed for each number in this list.
                
                Required input.
    \item[{type}]\hfill
\\The injection type for the disinfectant at the booster stations. The option is FLOWPACED. 
                See the EPANET 2.00.12 user manual for additional information about source types \cite{EPANETusermanual}.
                
                Required input.
    \item[{strength}]\hfill
\\The amount of disinfectant injected into the network from the booster stations. 
                For the source type FLOWPACED, the strength units are in mg/L.
            
                Required input.
    \item[{response time}]\hfill
\\The time in minutes between the detection of a contamination incident and 
                the start of injecting disinfectants from the booster stations. The value 
                is a nonnegative integer. For example, a value of 120 represents 
                a 120 minutes or a 2 hour delay between the time of detection and 
                the start of booster injections.
                
                Required input.
    \item[{duration}]\hfill
\\The length of time in minutes that disinfectant will be injected at the booster 
                stations during the simulation.	The value is a nonnegative integer. For example, 
                a value of 240 means that a booster would simulate injection of disinfectant 
                at a particular node for 4 hours. This duration is applied to all booster 
                station locations identified in the optimization process.
                
                Required input.
  \end{description}
  \item[{solver}]\hfill
  \begin{description}[topsep=0pt,parsep=0.5em,itemsep=-0.4em]
    \item[{type}]\hfill
\\The solver type. Each component of WST
				(e.g., sensor placement, flushing response, booster 
				placement) has different 
				solvers available. More specific details are provided in 
				the subcommand's chapter.
                
                Required input.
    \item[{options}]\hfill
\\A list of options associated with a specific solver type. More
            information on the options available for a specific solver
            is provided in the solver's documentation. The Getting
            Started Section \ref{dependencies} provides links to the
            different solvers.
            
            Optional input.
    \item[{threads}]\hfill
\\The maximum number of threads or function evaluations the solver is
                allowed to use.  This option is not available to all solvers or all analyses.
                
                Optional input.
    \item[{logfile}]\hfill
\\The name of a file to output the results of the solver.
                
                Optional input.
    \item[{verbose}]\hfill
\\The solver verbosity level.
                
                Optional input, default = 0 (lowest level).
    \item[{initial points}]\hfill
    \begin{description}[topsep=0pt,parsep=0.5em,itemsep=-0.4em]
      \item[{nodes}]\hfill
\\A list of node locations (EPANET IDs) to begin the optimization
        process. Currently, this option is only supported for the
        network solver used in the flushing and booster\_msx
        subcommands. This input causes an error for other subcommands.
        
        Optional input.
      \item[{pipes}]\hfill
\\A list of pipe locations (EPANET IDs) to begin the optimization
        process. Currently, this option is only supported for the
        network solver used in the flushing subcommand. This input causes an error for other subcommands.
        
        Optional input.
    \end{description}
  \end{description}
  \item[{configure}]\hfill
  \begin{description}[topsep=0pt,parsep=0.5em,itemsep=-0.4em]
    \item[{output prefix}]\hfill
\\The prefix used for all output files.
                
                Required input.
    \item[{output directory}]\hfill
      \\The output directory to store the results.
    \item[{debug}]\hfill
\\The debugging level (0 or 1) that indicates the amount of debugging 
                information printed to the screen, log file and output yml file. 
                
                Optional input, default = 0 (lowest level).
  \end{description}
\end{description}
