\begin{description}[topsep=0pt,parsep=0.5em,itemsep=-0.4em]
  \item[{network}]\hfill
  \begin{description}[topsep=0pt,parsep=0.5em,itemsep=-0.4em]
    \item[{epanet file}]\hfill
\\ The name of the EPANET 2.00.12 input (INP) file that defines the water distribution
                network model.
                
                Required input.
  \end{description}
  \item[{scenario}]\hfill
  \begin{description}[topsep=0pt,parsep=0.5em,itemsep=-0.4em]
    \item[{location}]\hfill
\\A list that describes the injection locations for the contamination scenarios.
                The options are: (1) ALL, which denotes all nodes (excluding tanks and reservoirs)
                as contamination injection locations; (2) NZD, which denotes all nodes with
                non-zero demands as contamination injection locations; or (3) an EPANET node ID, 
                which identifies a node as the contamination injection location. This allows 
                for an easy specification of single or multiple contamination scenarios.
                
                Required input unless a TSG or TSI file is specified.
    \item[{type}]\hfill
\\The injection type for the contamination scenarios. The options are MASS, CONCEN, FLOWPACED or SETPOINT. 
                See the EPANET 2.00.12 user manual for additional information about source types \citep{EPANETusermanual}.
                
                Required input unless a TSG or TSI file is specified.
    \item[{strength}]\hfill
\\The amount of contaminant injected into the network for the contamination scenarios.  
                If the type option is MASS, then the units for the strength are in mg/min. 
                If the type option is CONCEN, FLOWPACED or SETPOINT, then units are in mg/L.
                
                Required input unless a TSG or TSI file is specified.
    \item[{species}]\hfill
\\The name of the contaminant species injected into the network. This is the name of a single species. 
                It is required when using EPANET-MSX, since multiple species might be simulated, but
                only one is injected into the network. For cases where multiple contaminants are injected,
                a TSI file must be used.
                
                Required input for EPANET-MSX unless a TSG or TSI file is specified.
    \item[{start time}]\hfill
\\The injection start time that defines when the contaminant injection begins. 
                The time is given in minutes and is measured from the start of the simulation. 
                For example, a value of 60 represents an injection that starts at hour 1 of the simulation.
                
                Required input unless a TSG or TSI file is specified.
    \item[{end time}]\hfill
\\The injection end time that defines when the contaminant injection stops.				
                The time is given in minutes and is measured from the start of the simulation.
                For example, a value of 120 represents an injection that ends at hour 2 of the simulation.
                
                Required input unless a TSG or TSI file is specified.
    \item[{tsg file}]\hfill
\\The name of the TSG scenario file that defines the ensemble of contamination
                scenarios to be simulated. Specifying a TSG file will
                override the location, type, strength, species, start and end times options specified in
                the WST configuration file. The TSG file format is documented in File Formats Section \ref{formats_tsgFile}.
                
                Optional input.
    \item[{tsi file}]\hfill
\\The name of the TSI scenario file that defines the ensemble of contamination
                scenarios to be simulated. Specifying a TSI file will
                override the TSG file, as well as the location, type, strength, species, start and end time options specified in
                the WST configuration file. The TSI file format is documented in File Formats Section \ref{formats_tsiFile}.
                
                Optional input.
    \item[{signals}]\hfill
\\Name of file or directory with information to generate 
                or load signals. If a file is provided, the list of INP-TSG tuples
                 will be simulated and the information stored in signals files. If
                a directory with the signals files is specified, the signal files will
                be read and loaded in memory. This input is only valid for the uq
                subcommand and the grabsample subcommand with probability based formulations.

                Optional input.
    \item[{msx file}]\hfill
\\The name of the EPANET-MSX multi-species file that defines the multi-species reactions to
                be simulated using EPANET-MSX.
                
                Required input for EPANET-MSX.
    \item[{msx species}]\hfill
\\The name of the MSX species whose concentration profile will be saved by the EPANET-MSX simulation
                and used for later calculations.
                
                Required input for EPANET-MSX.
    \item[{merlion}]\hfill
\\A flag to indicate if the Merlion water quality
                simulator should be used. The options are true or false. 
                If an MSX file is provided, EPANET-MSX will be used.
                
                Required input, default = false.
  \end{description}
  \item[{grabsample}]\hfill
  \begin{description}[topsep=0pt,parsep=0.5em,itemsep=-0.4em]
    \item[{model format}]\hfill
\\The modeling language used to build the formulation specified
                by the model format option. The options are AMPL and PYOMO. 
                AMPL is a third party package that must be installed by 
                the user if this option is specified. PYOMO is an open source 
                software package that is distributed with WST.
       
                Required input, default = PYOMO.
    \item[{sample criteria}]\hfill
\\ Determines which optimization model to solve. This option is 
                only checked when running the problem with signal files. By default
                the optimization is based on distinguishability of pair-wise scenarios.
       
                Optional input.
    \item[{sample time}]\hfill
\\The time at which the manual grab sample should be taken. 
                The algorithm determines the best possible manual grab sample location(s)
                based upon this time. Units: Minutes from the simulation start time in the
                EPANET 2.00.12 INP file. 

                Required input.
    \item[{threshold}]\hfill
\\This threshold determines whether or not an incident impacts a candidate
                sample location.

                Required input, default = 0.001.
    \item[{fixed sensors}]\hfill
\\A list that defines nodes that are already fixed continuous sensor locations.
                The options are: (1) ALL, which specifies all nodes as fixed sensor locations;
                (2) NZD, which specifies non-zero demand nodes as fixed sensor locations;
                (3) NONE, which specifies no nodes as fixed sensor locations;
                (4) a list of EPANET node IDs, which identifies specific nodes as fixed sensor locations; or
                (5) a filename, which references a space or comma separated file containing a list of 
                specific nodes as fixed sensor locations. 

                Optional input.
    \item[{nodes metric}]\hfill
\\ File containing a map of node to metric. The map is used for determining weighting factors
                in the objective of the distinguishability optimization formulation.
                Each line in the file has the node name separated by the corresponding metric.  

                Optional input.
    \item[{list scenario ids}]\hfill
\\ File containing list of scenarios to considered from the signals folder.
                Each line in the file has the signals ID and the contamination ID separated by a space.
				
                Optional input.
    \item[{feasible nodes}]\hfill
\\A list that defines nodes that can be considered as potential sampling locations 
                for the optimal sample location problem.
                The options are: (1) ALL, which specifies all nodes as feasible sampling locations;
                (2) NZD, which specifies all non-zero demand nodes as feasible sampling locations;
                (3) a list of EPANET node IDs, which identifies specific nodes as feasible sampling locations; or
                (4) a filename, which references a space or comma separated file containing a list of 
                specific nodes as feasible sampling locations. 

                Optional input.
    \item[{num samples}]\hfill
\\The maximum number of locations that can be sampled at one time. This is usually equal
                to the number of sampling teams that are available.

                Required input, default = 1.
    \item[{greedy selection}]\hfill
\\The option to select manual grab sample locations based upon a greedy search, which orders and selects the locations in order of the best solution.
                This does not require any optimization.

                Optional input, default = false.
    \item[{with weights}]\hfill
\\The option to add weights in the objective function of the distinguishability 
                optimization formulation.

                Optional input, default = false.
    \item[{filter scenarios}]\hfill
\\ This option enables filtering scenarios. Only those scenarios 
                that match at least one of the measurements are considered
                in the optimal sampling analysis.
				
				Optional input, default = false.
  \end{description}
  \item[{solver}]\hfill
  \begin{description}[topsep=0pt,parsep=0.5em,itemsep=-0.4em]
    \item[{type}]\hfill
\\The solver type. Each component of WST
				(e.g., sensor placement, flushing response, booster 
				placement) has different 
				solvers available. More specific details are provided in 
				the subcommand's chapter.
                
                Required input.
    \item[{options}]\hfill
\\A list of options associated with a specific solver type. More
            information on the options available for a specific solver
            is provided in the solver's documentation. The Getting
            Started Section \ref{dependencies} provides links to the
            different solvers.
            
            Optional input.
    \item[{threads}]\hfill
\\The maximum number of threads or function evaluations the solver is
                allowed to use.  This option is not available to all solvers or all analyses.
                
                Optional input.
    \item[{logfile}]\hfill
\\The name of a file to output the results of the solver.
                
                Optional input.
    \item[{verbose}]\hfill
\\The solver verbosity level.
                
                Optional input, default = 0 (lowest level).
    \item[{initial points}]\hfill
    \begin{description}[topsep=0pt,parsep=0.5em,itemsep=-0.4em]
      \item[{nodes}]\hfill
\\A list of node locations (EPANET IDs) to begin the optimization
        process. Currently, this option is only supported for the
        network solver used in the flushing and booster\_msx
        subcommands. This input causes an error for other subcommands.
        
        Optional input.
      \item[{pipes}]\hfill
\\A list of pipe locations (EPANET IDs) to begin the optimization
        process. Currently, this option is only supported for the
        network solver used in the flushing subcommand. This input causes an error for other subcommands.
        
        Optional input.
    \end{description}
  \end{description}
  \item[{configure}]\hfill
  \begin{description}[topsep=0pt,parsep=0.5em,itemsep=-0.4em]
    \item[{output prefix}]\hfill
\\The prefix used for all output files.
                
                Required input.
    \item[{output directory}]\hfill
      \\The output directory to store the results.
    \item[{debug}]\hfill
\\The debugging level (0 or 1) that indicates the amount of debugging 
                information printed to the screen, log file and output yml file. 
                
                Optional input, default = 0 (lowest level).
  \end{description}
\end{description}
