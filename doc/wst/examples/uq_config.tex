\begin{description}[topsep=0pt,parsep=0.5em,itemsep=-0.4em]
  \item[{scenario}]\hfill
  \begin{description}[topsep=0pt,parsep=0.5em,itemsep=-0.4em]
    \item[{signals}]\hfill
\\Name of file or directory with information to generate 
                or load signals. If a file is provided the list of inp tsg tuples
                 will be simulated and the information stored in signals files. If
                a directory with the signals files is specified, the signal files will
                be read and loaded in memory. This input is only valid for the uq
                subcommand and the grabsample subcommand with probability based formulations.

                Required input.
  \end{description}
  \item[{uq}]\hfill
  \begin{description}[topsep=0pt,parsep=0.5em,itemsep=-0.4em]
    \item[{analysis time}]\hfill
\\The time at which the manual grab sample should be taken. 
                The algorithm determines the best possible manual grab sample location(s)
                based upon this time. Units: Minutes from the simulation start time in the
                EPANET 2.00.12 INP file.
    \item[{threshold}]\hfill
      \\This threshold determines whether or not an incident impacts a
      location (mg/L).
    \item[{filter scenarios}]\hfill
\\ This options enables filtering scenarios. Only those scenarios 
                that match at least one of the measurements are considered
                in the optimal sampling analysis, default = False.
    \item[{measurement failure}]\hfill
\\The probability that a sensor gives an incorrect reading. Must be between 0 and 1. 
                
                Required input for the Bayesian algorithm, default = 0.05.
    \item[{confidence}]\hfill
\\The probability cut-off value for classifying nodes as certain to be contaminated, 
                uncertain to be contaminated and certain to not be contaminated. The value is 
                between 0 and 1. Nodes with probability greater than (((1-confidence)/2)+confidence) are 
                classified likely to be contaminated or likely yes LY, 
                nodes with probability less than ((1-confidence)/2) are classified likely not contaminated
                LN and nodes with probability in between are uncertain nodes UN.
                
                Required for node classification, default = 0.95 (unitless).
  \end{description}
  \item[{measurements}]\hfill
  \begin{description}[topsep=0pt,parsep=0.5em,itemsep=-0.4em]
    \item[{grab samples}]\hfill
\\The name of the file that contains all the measurements from 
                the manual grab samples and the fixed sensors. The measurement file 
                format is documented in File Formats Section \ref{formats_measFile}.

                Optional input.
  \end{description}
  \item[{configure}]\hfill
  \begin{description}[topsep=0pt,parsep=0.5em,itemsep=-0.4em]
    \item[{output prefix}]\hfill
\\The prefix used for all output files.
                
                Required input.
    \item[{output directory}]\hfill
      \\The output directory to store the results.
    \item[{debug}]\hfill
\\The debugging level (0 or 1) that indicates the amount of debugging 
                information printed to the screen, log file and output yml file. 
                
                Optional input, default = 0 (lowest level).
  \end{description}
\end{description}
