In July of 1970, the White House and Congress worked together to establish 
the United States Environmental Protection Agency (EPA) in response to the 
growing public demand for cleaner water, air, and land. The Agency was assigned 
the daunting task of repairing the damage already done to the natural 
environment and establishing new criteria to guide Americans in making a cleaner 
environment a reality. Since 1970, EPA has worked with federal, state, tribal, 
and local partners to advance its mission to protect human health and the 
environment.

EPA leads the nation's environmental science, research, education, and assessment 
efforts. With more than 17,000 employees across the country, EPA works to 
research, develop, and enforce regulations that implement environmental laws 
enacted by Congress. In recent years, between 40 and 50 percent of EPA's 
enacted budgets have provided direct support through grants to State 
environmental programs. At laboratories located throughout the nation, the 
Agency works to assess environmental conditions and to identify, understand, and 
solve current and future environmental problems. The Agency works through its 
headquarters and regional offices with over 10,000 industries, businesses, 
nonprofit organizations, and state and local governments, on over 40 
voluntary pollution prevention programs and energy conservation efforts. 

Under existing laws and recent Homeland Security Presidential Directives, EPA 
has been called upon to play a vital role in helping to secure the nation 
against foreign and domestic enemies. The National Homeland Security Research 
Center (NHSRC) was formed in 2002 to conduct research in support of EPA's role 
in homeland security. NHSRC research efforts focus on five areas: water 
infrastructure protection, threat and consequence assessment, decontamination 
and consequence management, response capability enhancement, and homeland 
security technology testing and evaluation. EPA is the lead federal agency 
for drinking water and waste water systems and NHSRC is working to reduce 
system vulnerabilities, prevent and prepare for terrorist attacks, minimize 
public health impacts and infrastructure damage, and enhance recovery efforts. 

This \docTitle\ for the Water Security Toolkit software package is published 
and made available by EPA's Office of Research and Development to assist the 
water community in improving the security of our nation's drinking water.

\vspace*{\baselineskip}
\vspace*{\baselineskip}

\begin{tabular}{ll}
 & Gregory Sayles, Ph.D., Acting Director \\

& National Homeland Security Research Center \\
& Office of Research and Development \\
& U. S. Environmental Protection Agency
\end{tabular}
