This chapter describes the different file formats used by WST, including a 
brief description, format, the associated subcommand(s) and any additional details.

\section{Configuration File}\label{formats_yamlFile}
\begin{itemize}
\item {\bfseries Description:} Input configuration file for all WST subcommands.
\item {\bfseries Format:} YAML 
\item {\bfseries Created by:} Template input configuration files can be created using the \code{---template} option from each subcommand in WST.
\item {\bfseries Used by:} WST
\item {\bfseries Details:} 
The input configuration files for WST are stored in the YAML file format.
YAML is a human-readable file format that is well suited for storing hierarchical 
information. This information can be easily parsed and stored as common data types 
such as strings, scalars, lists and dictionaries by a range of programming languages. 
WST uses PyYAML to parse YAML files into Python data types. Basic YAML format specifications are listed below:
\begin{itemize}
\item Each element of a YAML file is a key, value pair separated by a colon (key:value).  
\item The key is the name given to the element, and the value is the data for that element.  
\item The hierarchy of YAML files is maintained by outline indentation. 
\item The number of spaces used to indent an element in the YAML file must be consistent across all elements at the same hierarchical level.  
\item Nested data elements must be indented further than their preceding level.
\item Using tab for indentation is not recommended.
\item Optional blank lines can be added for readability. 
\item Comments begin with the number sign (\#) and must be separated from a key:value 
pair by space. Comments can start anywhere but are limited to a single line.
\item PyYAML automatically casts data types. For example, [123] is read as a list 
with a single integer value, '123' is read as a string, 123 is read as an integer, and 123.0 is read as a real number.
\item Strings do not require quotation (unless they could be cast as a number) and can contain spaces.
\item Lists are indicated with square brackets or hyphens. When using square brackets, the list is comma separated. When using hyphens, each entry of the list is on a new line.
\item Dictionaries are indicated with indentation or curly brackets and are used to define key: value pairs.  Nested dictionaries define the hierarchical levels in the YAML file.
\end{itemize}
Additional information on YAML files can be found on the official YAML website \url{http://www.yaml.org}.

Select aspects of the \code{flushing} subcommand template configuration file are used as an example of the format of a YAML file. 
The full \code{flushing} subcommand template configuration file is shown in Figure \ref{fig:flushing_template}. 
In the template, the top level key, denoted 'flushing', contains the following data:
\unknownInputListing{examples/flushing_config.yml}{}{1}{50}
This subset of the the \code{flushing} subcommand template is refereed to as the flushing block. Instead of a single 
value assigned to 'flushing', the value is a dictionary containing a nested structure 
of additional key:value pairs. 

The keys 'detection', 'flush nodes' and 'close valves' are all second level keys inside the flushing block. 
The location of these keys is often specified using the notation [flushing][detection], [flushing][flush nodes] and [flushing][close valves]. 
The second level keys must be indented using the same number of spaces and they must have unique names. 
The key [flushing][detection] is assigned to a list. Lists can be specified 
in one of two ways, using square brackets or using hyphen. The following two formats (separated by -\--\--) are equivalent. 
\begin{unknownListing}
flushing:
  detection: [111, 127, 179] # square bracket notation, comma separated
---
flushing:
  detection: # hyphen notation, new line for each entry
  - 111
  - 127
  - 179
\end{unknownListing}
All template configuration files use square brackets to indicate where a list of input values can be used.
If these input values include keywords, like NZD for non-zero demand nodes, this 
information is listed in the comment or in the user manual documentation for 
that specific YAML input parameter.

The options [flushing][flush nodes] and [flushing][close valves] are both assigned dictionaries. 
The data inside these second level keys contain additional key:value pairs. 
All keys within these dictionaries must be indented using the same number of spaces and have unique names. 
Two key:value pairs in the flushing flush nodes option are listed below:
\begin{unknownListing}
flushing:
  flush nodes:
    feasible nodes: NZD
    max nodes: 2
\end{unknownListing}    
Here, the [flushing][flush nodes][feasible nodes] option is set to the string NZD 
and the [flushing][flush nodes][max nodes] option is set to the integer 2. 
Note that there are two third-order keys named 'response time', however they do not share the same exact hierarchy, as shown below:
\begin{unknownListing}
flushing:
  flush nodes:
    response time: 0.0
  close valves:
    response time: 0.0    
\end{unknownListing}    

YAML files can be written in a compact format that uses curly brackets to represent the hierarchical indentation of the nested dictonary. 
While this avoids issues with space indentation, this format is more difficult for the user to read. 
For example, the following two examples (separated by -\--\--) are equivalent: 
\begin{unknownListing} 
flushing:
  detection: [111, 127, 179] 
---
{'flushing': {'detection': [111, 127, 179]}}
\end{unknownListing}  
\end{itemize}

\section{Cost File}\label{formats_costFile}
\begin{itemize}
\item {\bfseries Description:} Specifies the costs for installing sensors at 
different nodes throughout a network. This is the cost of installing one sensor 
at one particular node. 
\item {\bfseries Format:} ASCII 
\item {\bfseries Created by:} WST user 
\item {\bfseries Used by:} \code{sp} 
\item {\bfseries Details:} Each line of this file has the format:
\begin{unknownListing}
   <EPANET node ID> <cost>
\end{unknownListing}
Nodes not explicitly enumerated in this file are assumed to have a cost of zero 
unless the ID \code{\_\_}default is specified. For example to specify that 
all un-\/enumerated nodes have a cost of 1.0:
\begin{unknownListing}
   __default 1.0
\end{unknownListing}

For example, the following cost file indicates that a sensor at node 1 has a cost of \$100, a sensor at node 2 has a cost of \$200 and sensors at all other nodes have a cost of \$50.
\begin{unknownListing}
     1       100
     2       200
   __default 50
\end{unknownListing}
\end{itemize}

\if 0
\section{DVF File}\label{formats_dvfFile} 
\begin{itemize} 
\item {\bfseries Description:} Provides representation of flushing variables (opened hydrants and closed pipes).
\item {\bfseries Format:} ASCII 
\item {\bfseries Created by:} WST user or \code{flushing}
\item {\bfseries Used by:} \code{tevasim} and \code{sim2Impact}
\item {\bfseries Details:}
\unknownInputListing{fileFormats/dvfFormat.txt}{}{1}{16}{}
\end{itemize} 
\fi

\section{ERD File}\label{formats_tsoFile} 
\begin{itemize} 
\item {\bfseries Description:} Provides a compact representation of all contamination scenario simulation results. 
\item {\bfseries Format:} binary 
\item {\bfseries Created by:} \code{tevasim} 
\item {\bfseries Used by:} \code{sim2Impact} 
\item {\bfseries Details:} The simulation data generator produces four output files 
containing the results of all contamination simulation scenarios.  The database files include 
an index file (index.erd), a hydraulics file (hyd.erd) and a water quality file (qual.erd).  
The files are unformatted binary file in order to save disk space and computation time. 
They are not readable using an ordinary text editor.  
%All data written are either of type float, int, long, or 16-byte character strings. The format of this file is as follows. 
%\unknownInputListing{fileFormats/tsoFormat.txt}{}{}
\item {\bfseries Note:} The ERD file format replaced the TSO and SDX file formats, created by previous versions of 
\code{tevasim}, to extend the capability of \code{tevasim} for multi-species simulation using EPANET-MSX.  
While the \code{tevasim} subcommand produces only ERD files 
(even for single species simulation), the \code{sim2Impact} subcommand accepts both ERD and TSO file formats. 
\end{itemize} 


\section{Impact File}\label{formats_impactFile}
\begin{itemize}
\item {\bfseries Description:} For each contamination scenario, the impact file 
contains a list of all the locations (nodes) in the network where a sensor might 
detect contamination from a specific scenario.
\item {\bfseries Format:} ASCII 
\item {\bfseries Created by:} \code{sim2Impact} 
\item {\bfseries Used by:} \code{sp} and \code{evalsensor} 
\item {\bfseries Details:} The first line of an impact file 
contains the number of incidents. The second line specifies the number of delays (always 1) and the delay time in minutes.
Subsequent lines have the format
\begin{unknownListing}
   <scenario-index> <node-index> <time-of-detection> <impact-value>
\end{unknownListing}
The scenario index is the index of contamination scenarios that were simulated. 
A scenario index maps to a line in the network scenariomap file, which is defined in Section \ref{formats_scenarioFile}. 
The node index is the index of a witness location for the incident. A node index maps to a line 
in the network nodemap file, which is defined in Section \ref{formats_nodeFile}. The time of detection is in minutes. 
The value of the impacts are in the corresponding units for each impact metric. The different 
impact metrics in each line correspond to the different delays that have been computed. 
\end{itemize}


\section{Imperfect Junction Class File}\label{formats_junctionClass}
\begin{itemize}
\item {\bfseries Description:} Provides the mapping from EPANET node IDs to failure classes of different false-negative probabilities. 
\item {\bfseries Format:} ASCII 
\item {\bfseries Created by:} WST user 
\item {\bfseries Used by:} \code{sp} 
\item {\bfseries Details:} The imperfect junction class file provides the mapping from EPANET node IDs
 to failure classes of different false-negative probabilities as defined in a imperfect sensor class file 
 (See Section \ref{formats_sensorClass} for information on imperfect sensor class files). The format of this file is:
\begin{unknownListing}
   <node id> <failure class>
   <node id> <failure class>
   ....
\end{unknownListing}

For example, node 1 is of class 2, node 2 is of class 1 and node 3 is of class 1:
\begin{unknownListing}
   1 2
   2 1
   3 1
   ....
\end{unknownListing}
\end{itemize}

\section{Imperfect Sensor Class File}\label{formats_sensorClass}
\begin{itemize}
\item {\bfseries Description:} Contains false-negative probabilities for different types of sensors. 
The false-negative probability defines the accuracy rate of the sensor (e.g., 50 percent of the 
time the sensor is providing a correct reading). 
\item {\bfseries Format:} ASCII 
\item {\bfseries Created by:} WST user 
\item {\bfseries Used by:} \code{sp} 
\item {\bfseries Details:} The file has format:
\begin{unknownListing}
   <class id> <false-negative probability>
   <class id> <false-negative probability>
   ....
\end{unknownListing}

For example, the following file defines a failure class 1 with a false-negative probability of 25 percent, 
and a failure class 2 with a false-negative probability of 50 percent:
\begin{unknownListing}
   1 0.25
   2 0.5
   ....
\end{unknownListing}
\end{itemize}

\section{Measurements File}\label{formats_measFile}
\begin{itemize}
\item {\bfseries Description:} Contains a list of measurements along with their corresponding time and EPANET node ID.
  This file can contain multiple node IDs and the measurement time is not required to be in order. 
\item {\bfseries Format:} ASCII
\item {\bfseries Created by:} WST user or a water quality event detection system or a data acquisition system  
\item {\bfseries Used by:} \code{inversion} 
\item {\bfseries Details:} Each line of this file has the format:
\begin{unknownListing}
   <EPANET node ID> <Time from beginning of simulation (sec) > <Binary yes/no measurement>
\end{unknownListing}
An example measurements file is provided:
\unknownInputListingFixed{../../examples/Net3/Net3_MEASURES.dat}{}{1}{10}
\end{itemize}

\section{Nodemap File}\label{formats_nodeFile}
\begin{itemize}
\item {\bfseries Description:} Provides a mapping from the indices used for sensor placement to the node IDs used within EPANET.
\item {\bfseries Format:} ASCII 
\item {\bfseries Created by:} \code{sim2Impact} 
\item {\bfseries Used by:} \code{evalsensor} and \code{sp} 
\item {\bfseries Details:} Each line of this file has the format:
\begin{unknownListing}
   <node-index> <EPANET node ID>
\end{unknownListing}
This mapping is generated by the \code{sim2Impact} subcommand, and all sensor placement solvers subsequently use the node indices internally. 
\end{itemize}


\section{Scenariomap File}\label{formats_scenarioFile}
\begin{itemize}
\item {\bfseries Description:} The scenariomap file provides auxiliary information about each contamination incident. 
\item {\bfseries Format:} ASCII 
\item {\bfseries Created by:} \code{sim2Impact} 
\item {\bfseries Used by:} \code{evalsensor} 
\item {\bfseries Details:} Each line of this file has the format:
\begin{unknownListing}
   <node-index> <EPANET node ID> <source-type> <source-start-time> ...
			<source-stop-time> <source-strength>
\end{unknownListing}
The node index maps to the network nodemap file as described in Section \ref{formats_nodeFile}, and the EPANET node ID provides 
this information. The source type is the injection mode for an scenario, e.g., 
flow-\/paced or fixed-\/concentration. The scenario source start and stop times are in minutes, and these values are relative to the 
start of the EPANET simulation. The source strength is the concentration of contaminant at the injection source. 
\end{itemize}


\section{Sensor Placement File}\label{formats_sensorPlacementFile}
\begin{itemize}
\item {\bfseries Description:} Describes one or more sensor placements. 
\item {\bfseries Format:} ASCII 
\item {\bfseries Created By:} \code{sp} 
\item {\bfseries Used By:} \code{evalsensor} 
\item {\bfseries Details:} Lines in a sensor placement file that begin with 
the \# character are assumed to be comments. Otherwise, lines of this file have the format
\begin{unknownListing}
   <sp-id> <number-of-sensors> <node-index-1> ...
\end{unknownListing}
The sensor placement ID is used to identify sensor placements in the file. Sensor 
placements could have differing numbers of sensors, so each line contains this 
information. The node indices map to values in the nodemap file described in Section \ref{formats_nodeFile}. 
\end{itemize}

\section{TAI File}\label{formats_taiFile}
\begin{itemize}
\item {\bfseries Description:} Describes the information needed for assessing health impacts.
\item {\bfseries Format:} ASCII 
\item {\bfseries Created by:} WST user 
\item {\bfseries Used by:} \code{sim2Impact} 
\item {\bfseries Details:} This file is required for health impact metrics, such as population exposed, population 
dosed and population killed. 
The following example can be copied directly into a text editor. 
\unknownInputListing{fileFormats/tiaFormat.txt}{1}{1}{102}
\end{itemize}


\section{TSG File}\label{formats_tsgFile}
\begin{itemize}
\item {\bfseries Description:} Specifies how an ensemble of EPANET 2.00.12 contamination scenario simulations will be performed. 
\item {\bfseries Format:} ASCII 
\item {\bfseries Created by:} WST user 
\item {\bfseries Used by:} \code{tevasim} 
\item {\bfseries Details:} Each line of a TSG file specifies injection location(s), species (optional), 
injection mass and the injection time-\/frame:
\begin{unknownListing}
   <injection-location> <injection-type> <specie> <injection-mass> <start-time> <end-time>
\end{unknownListing}	
If <specie> is included, the \code{tevasim} subcommand uses EPANET-MSX. The simulation data generator 
uses the specifications in the TSG file to construct a separate threat simulation input (TSI) 
file that describes each individual contamination scenario in the ensemble. Each line in the TSG file 
uses a simple language that is expanded to define the ensemble. The entire ensemble is 
comprised of the cumulative effect of all lines in the TSG file. The TSG file is an optional file, 
since the ensemble of contamination scenarios can be specified in the configuration file.
\unknownInputListing{fileFormats/tsgFormat_msx.txt}{}{1}{44}{}
\end{itemize}


\section{TSI File}\label{formats_tsiFile} 
\begin{itemize} 
\item {\bfseries Description:} Specifies how an ensemble of EPANET 2.00.12 contamination scenario simulations will be performed. 
\item {\bfseries Format:} ASCII 
\item {\bfseries Created by:} \code{tevasim} or WST user 
\item {\bfseries Used by:} \code{tevasim} 
\item {\bfseries Details:} The TSI file is generated as output from the \code{tevasim} subcommand 
and would not normally be used, but it is available after the run for reviewing each scenario 
that was generated for the ensemble. The TSG file is essentially a short hand for generation of 
the more cumbersome TSI file. Each record in the TSI file specifies the unique attributes 
of one contamination scenario. The number of scenarios does not have a restriction.                         
\unknownInputListing{fileFormats/tsiFormat_msx.txt}{1}{1}{23}
\end{itemize} 


\section{Weight File}\label{formats_weightFile}
\begin{itemize}
\item {\bfseries Description:} Specifies the weights for contamination incidents.
\item {\bfseries Format:} ASCII 
\item {\bfseries Created by:} WST user 
\item {\bfseries Used by:} \code{sp} 
\item {\bfseries Details:} Each line of this file has the format:
\begin{unknownListing}
   <scenario-ID> <weight>
\end{unknownListing}
Scenarios not explicitly enumerated in this file are assumed to have a weight of zero unless 
the ID \code{\_\_}default is specified. For example, to specify that all un-\/enumerated 
scenarios have a weight of 1.0:
\begin{unknownListing}
   __default 1.0
\end{unknownListing}
\end{itemize}




% LocalWords:  WST sp un
