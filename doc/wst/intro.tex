An abundant supply of safe, high-quality drinking water is critical to modern industrialized societies. 
At home, water is used for drinking, cooking, washing clothes and bathing. At work, water 
is used to operate restaurants, hospitals and manufacturing plants. In our communities, 
water is used for fighting fires. Consequently, contamination of drinking 
water infrastructure could severely impact the public health and economic vitality of a community. 
The distributed physical layout of drinking water systems makes them inherently vulnerable to a variety 
of incidents, such as terrorist attacks, accidents and even natural disasters. 
The physical destruction of water infrastructure can disrupt water service to communities; 
specifically key facilities such as hospitals, power stations and military installations. Similarly, 
contamination with deadly agents could result in large numbers of illnesses and fatalities.

Since the events of September 11, 2001, water utilities have had increasing concerns about 
the possibility of harm to our water quality due to an accidental or intentional contamination incident within a distribution network. 
The U.S. EPA's Response Protocol Toolbox \citep{ResponseProtocolToolbox} provides recommendations on actions that 
water utilities can take to minimize potential impacts to consumers following a contamination threat or incident. 
Detection and consequence management are major steps in this protocol. EPA has also developed modeling and 
simulation tools to assist in the detection of contamination incidents in water distribution networks. 
The Threat Ensemble Vulnerability Assessment-Sensor Placement Optimization Tool, or TEVA-SPOT \citep{TEVASPOTusermanual}, 
identifies the optimal placement of online water quality monitoring sensors to detect contamination incidents. 
Another EPA developed tool to assist in detection is the CANARY event detection system \citep{CANARY}, 
which analyzes water quality data from sensors and identifies periods of anomalous water quality. 
These tools work together to help form a contamination warning system (CWS). The overall goal of a 
CWS is to detect contamination incidents in time to reduce potential public health and economic consequences. 
The current terminology for a CWS is a water quality surveillance and response system. For more information 
on CWS, see U.S. EPA Water Security Initiative \citep{WSI}.

Should a CWS detect the presence of contamination in a water distribution network, consequence management must be employed. 
Decision-making tools that assist water utilities in evaluating and planning various response strategies 
are needed to support rapid response to contamination incidents. The Water Security Toolkit (WST) is a suite 
of tools that help provide the information necessary to make good decisions 
resulting in the minimization of further human exposure to contaminants, and the maximization of the 
effectiveness of intervention strategies. WST is intended to assist in: 
\begin{itemize}
\item Planning response actions to natural disasters and terrorist attacks, 
\item Developing consequence management plans, 
\item Informing large-scale exercises/training, 
\item Planning response actions to address traditional utility challenges, such as pipe breaks and water quality problems and 
\item Evaluating implications of different response strategies. 
\end{itemize}
For water utilities with hydraulic modeling expertise, WST combined with EPANET-RTX \citep{EPANETRTX,HatchettetalRTX2011,JankeetalRTX2011} 
could use data from CANARY, other sensor stations and field investigations to optimize and 
implement response actions in real-time. 

WST assists in the evaluation of multiple response actions in order to select the most 
beneficial consequence management strategy. It includes hydraulic and water quality 
modeling software and optimization methodologies to identify: (1) sensor locations to detect contamination,
(2) locations in the network in which the contamination was introduced, (3) hydrants to 
remove contaminated water from the distribution system, (4) locations in the network to inject 
decontamination agents to inactivate, remove or destroy contaminants and (5) locations in the network to take grab
sample to confirm contamination or cleanup.

This user manual describes the different components of WST. 
\ifEPAReport
It is also available as a Sandia Report (\citet{WSTSandreport}).
\else
It is also available as an EPA Report (\citet{WSTEPAreport}).
\fi
The manual contains one chapter on each of the water security tools:
\begin{itemize}
\item Contaminant transport
\item Impact assessment
\item Sensor placement
\item Hydrant flushing 
\item Booster placement
\item Source identification
\item Grab sampling
\item Visualization
\end{itemize}
Another chapter discusses advanced topics and provides case studies. 
WST uses YAML format configuration files to supply input parameters to each water security tool.  
Additional information on the YAML format can be found in File Formats Section \ref{formats_yamlFile}.

The contaminant transport simulation, impact assessment and sensor placement optimization tools were all 
developed as part of the TEVA-SPOT Toolkit \citep{TEVASPOTusermanual}. All functionality in TEVA-SPOT has been replicated 
in WST using new, user friendly YAML format configuration files. WST builds upon the simulation and optimization 
framework of TEVA-SPOT and adds several new features. These features were all 
developed to model possible response action plans once a contamination incident has been detected in the system. 
These action plans include redirecting flow by opening hydrants and closing valves, injecting decontaminant 
to inactivate biological agents and using sensor measurements to identify possible source locations.

The main data requirement to use WST is a calibrated water utility network model. Additional input data is 
dependent on the WST application. This includes information on the simulated contamination incident(s) 
(e.g., type, location(s), amount), the impact metric (e.g., extent of contamination, population exposed) 
and the response actions (e.g., flushing hydrants, injecting disinfectant). To optimize a response action, 
WST must be given additional information about the potential locations for water quality sensors, hydrants to flush, valves to close, 
disinfectant booster stations and manual grab samples. The operating characteristics of these different response 
actions are also required, such as the detection limits of the water quality sensors, the rate and duration that 
hydrants can be flushed, the control settings for injecting disinfectant at booster stations and the number of manual 
grab samples that can be taken at the same time. More details on the data requirements are provided 
in the chapter describing each of the specific water security tool. In addition, each chapter has example applications. 
All examples are included with WST and can be found in the examples folder. These examples use simple networks 
and data files that are also distributed with WST. The examples shown in this user manual are all executed on a 
Linux computer, so the CPU time for each example might not be the same on computers with different operating systems. 
